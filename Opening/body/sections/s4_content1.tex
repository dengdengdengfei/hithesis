\subsection{主要研究内容}
(撰写宜使用将来时态,不能只列出论文目录来代替对研究内容的分析论述)



% ---------------------------------------------------------------------
% 4.1 主要研究内容
% ---------------------------------------------------------------------


% 定义流程图样式
\tikzstyle{process} = [rectangle, minimum width=3cm, minimum height=1cm, text centered, draw=black, fill=blue!10, rounded corners, align=center]
\tikzstyle{decision} = [diamond, minimum width=2cm, minimum height=1cm, text centered, draw=black, fill=green!10, aspect=2]
\tikzstyle{data} = [trapezium, trapezium left angle=70, trapezium right angle=110, minimum width=2cm, minimum height=1cm, text centered, draw=black, fill=yellow!10]
\tikzstyle{startstop} = [rectangle, rounded corners, minimum width=3cm, minimum height=1cm, text centered, draw=black, fill=red!10, align=center]
\tikzstyle{arrow} = [thick,->,>=Stealth]
\tikzstyle{bidiarrow} = [thick,<->,>=Stealth]





本研究围绕"如何理解和分析形式不具独特性却具有强烈类型身份的建筑"这一核心问题,从理论建构、知识表征、形式分析、规则解析四个层面展开,形成"理论引领—方法支撑—实证验证"的整体研究结构。

\subsubsection{罗西类型学的理论拓展:规则驱动型建筑的类型学建构}

本部分是全文的理论基石,旨在回应罗西类型学面对"形式模糊型"建筑时的解释力不足问题,提出适用于中国传统民间建筑的类型学补充理论。

\paragraph{罗西类型学的批判性重读}

系统梳理罗西类型学的三个核心概念——"类型"(Type)、"持久性"(Permanence)与"记忆装置"(Apparatus),辨析其理论预设与方法论特征。重点考察以下问题:

\begin{itemize}[leftmargin=2em]
    \item 罗西如何超越迪朗的形式分类传统,将类型重构为"集体记忆的载体"?
    \item "推进性持久性"与"病理性持久性"的区分对理解建筑演化有何启示?
    \item "记忆装置"的隐喻蕴含了怎样的认识论假设?其操作化路径为何缺失?
\end{itemize}

在此基础上,明确罗西类型学的适用边界:其核心洞见依赖于"类型的可识别性源于形式的独特性与持久性"这一隐含假设,而这一假设主要来自西方城市与纪念碑建筑的经验,对于形式与普通民居高度趋同的中国传统宗祠建筑难以直接适用。

\paragraph{"规则驱动型建筑"的概念建构}

针对罗西类型学的理论空白,提出"规则驱动型建筑"(Rule-driven Architecture)概念,并进行严格的概念界定:

\begin{itemize}[leftmargin=2em, label={}]
    \item \textbf{定义:}一类形态生成主要受社会规则(礼制、禁忌、等级、仪式程序)支配,而非纯粹形式演化逻辑或功能需求主导的建筑类型。其类型性不体现为视觉形式的独特性,而体现为"规则-形式转译结构"的稳定性。
    
    \item \textbf{与"形式驱动型建筑"的对照:}
\end{itemize}

\begin{table}[h]
\centering
\caption{形式驱动型建筑与规则驱动型建筑对照}
\begin{tabular}{p{3cm}p{5cm}p{5cm}}
\toprule
\textbf{维度} & \textbf{形式驱动型建筑} & \textbf{规则驱动型建筑} \\
\midrule
典型案例 & 哥特教堂、希腊神庙 & 中国宗祠、日本茶室 \\
类型识别依据 & 视觉形式的独特性(尖拱、柱式) & 转译结构的稳定性(礼制编码) \\
形式与功能关系 & 形式具有符号自足性 & 形式是规则的空间编码 \\
意义解读路径 & 形式 → 意义 & 形式 → 规则 → 意义 \\
\bottomrule
\end{tabular}
\end{table}

\begin{itemize}[leftmargin=2em, label={}]
    \item \textbf{概念的理论定位:}规则驱动型建筑不是对形式驱动型建筑的否定,而是类型学光谱上的另一端点。大多数建筑处于二者之间,但某些建筑类型明显偏向规则驱动端,广府宗祠即为典型。
\end{itemize}

\paragraph{对罗西核心概念的补充与操作化}

基于"规则驱动型建筑"概念,对罗西的"持久性"与"记忆装置"概念进行补充和操作化:

\begin{itemize}[leftmargin=2em, label={}]
    \item \textbf{对"持久性"的补充:}罗西论证的持久性是"形式持久,功能流变"。本研究补充命题:对于规则驱动型建筑,\textbf{转译结构持久,形式可变}。广府宗祠的外观形式与民居趋同(形式层不具独特性),但"礼制规则→空间配置"的转译逻辑跨越数百年保持稳定。类型的持久性从"形式层"转移至"转译结构层"。
    
    \item \textbf{对"记忆装置"的操作化:}罗西将建筑比喻为记忆装置,但未说明记忆如何编码、如何解码。本研究将其操作化为:建筑是\textbf{规则的编码装置},其运作机制可分解为:
    \begin{enumerate}[label=\alph*., leftmargin=3em]
        \item \textbf{编码端:}社会规则通过特定转译机制物化为空间形式(等级→轴线位置,禁忌→边界设置,仪式→动线编排)
        \item \textbf{解码端:}观察者需还原规则才能理解形式的意义(从可见的形式回溯不可见的规则)
        \item \textbf{验证方式:}通过知识图谱形式化表达转译结构,通过RAG推理验证解码路径
    \end{enumerate}
\end{itemize}

\paragraph{配套分析方法的理论整合}

"规则驱动型建筑"理论需要配套的分析方法才能落地。本部分阐明符号学、空间句法与知识工程如何整合为连贯的方法体系:

\begin{itemize}[leftmargin=2em, label={}]
    \item \textbf{符号学的方法论贡献:}引入皮尔斯符号学的三元关系(符号-对象-解释项),将建筑空间分析重构为符号分析。空间元素是符号,礼制规则是对象,使用者/研究者的理解是解释项。这一框架提供了"编码-解码"的分析视角,弥合形式描述与意义解释之间的鸿沟。
    
    \item \textbf{空间句法的拓展方向:}空间句法的贡献在于将空间关系拓扑化、可计算化,但其核心局限在于预设"无身份的行人"——计算任何人从A到B的物理可达性,忽视社会规则对特定身份的空间约束。本研究提出"伦理句法"概念,引入"身份维度":计算\textbf{特定身份}从A到B的\textbf{规则许可性}。这要求将礼制规则形式化并嵌入空间分析模型。
    
    \item \textbf{知识图谱的方法论角色:}知识图谱作为"转译结构"的形式化表征工具,将隐性的规则-形式关系显性化、可计算化。核心建模要素包括:
    \begin{enumerate}[label=\alph*., leftmargin=3em]
        \item 实体类型:空间单元、礼制规则、宗族角色、仪式行为
        \item 关系类型:encode\_as(编码为)、restrict\_access(限制进入)、prescribe\_position(规定位置)、has\_rank(具有等级)
        \item 推理能力:基于规则进行情境判断(某身份在某情境下能否进入某空间)
    \end{enumerate}
\end{itemize}

% 方法整合流程图
\begin{figure}[h]
\centering
\begin{tikzpicture}[node distance=2cm]
    \node (semiotics) [process, align=center] {符号学框架\\(编码-解码)};
    \node (syntax) [process, below of=semiotics] {空间句法\\(伦理句法)};
    \node (kg) [process, below of=syntax] {知识图谱\\(转译结构)};
    \node (rule) [process, right=3cm of syntax] {规则驱动型建筑理论};
    
    \draw [arrow] (semiotics) -- node[anchor=west] {提供分析视角} (rule);
    \draw [arrow] (syntax) -- node[anchor=west] {计算身份约束} (rule);
    \draw [arrow] (kg) -- node[anchor=south] {形式化表征} (rule);
    
    \draw [bidiarrow] (semiotics) -- (syntax);
    \draw [bidiarrow] (syntax) -- (kg);
\end{tikzpicture}
\caption{配套分析方法的理论整合}
\end{figure}

\subsubsection{基于CIDOC CRM的岭南传统建筑知识库构建}

本部分是全文的数据基础设施,旨在将礼制规则、空间配置、宗族结构等异质知识整合为结构化、可计算的知识库,为后续分析提供支撑。

\paragraph{领域本体设计}

以CIDOC CRM(ISO 21127)为上层本体,设计岭南传统建筑领域本体。遵循以下设计原则:

\begin{itemize}[leftmargin=2em]
    \item \textbf{语义兼容性:}所有核心类均映射至CIDOC CRM,确保与国际遗产信息标准的互操作性
    \item \textbf{领域适切性:}扩展类与关系充分反映岭南传统建筑与礼制空间的特殊性
    \item \textbf{分析可用性:}本体结构服务于后续的形式分析与规则推理需求
\end{itemize}

核心类设计如下:

\begin{table}[h]
\centering
\caption{领域本体核心类设计}
\begin{tabular}{p{3.5cm}p{3.5cm}p{6cm}}
\toprule
\textbf{领域类} & \textbf{CIDOC CRM映射} & \textbf{说明与示例} \\
\midrule
LNA\_Building & E22 Man-Made Object & 祠堂建筑实体,如"陈家祠" \\
LNA\_SpatialUnit & E53 Place & 空间单元,如"正堂""天井""两廊""寝堂" \\
LNA\_ArchElement & E22 Man-Made Object & 建筑构件,如"神主龛""匾额""脊饰" \\
LNA\_ClanRole & E21 Person / E74 Group & 宗族角色,如"长房""嫡出""女眷""外姓" \\
LNA\_Ritual & E7 Activity & 仪式活动,如"春秋祭祀""丧礼""婚礼" \\
LNA\_RitualRule & E89 Propositional Object & 礼制规则,如"昭穆之序""男女之别" \\
LNA\_TranslationRel & P138 represents(扩展) & 规则→形式的编码关系 \\
\bottomrule
\end{tabular}
\end{table}

核心关系设计如下:

\begin{itemize}[leftmargin=2em]
    \item \textbf{空间关系:}\texttt{locate\_in}(位于)、\texttt{adjacent\_to}(相邻)、\texttt{axially\_aligned}(轴线对齐)、\texttt{hierarchically\_above}(等级高于)
    \item \textbf{规则关系:}\texttt{governs}(支配)、\texttt{restricts\_access\_for}(限制……进入)、\texttt{prescribes\_position\_for}(规定……位置)
    \item \textbf{转译关系:}\texttt{encoded\_as}(编码为)、\texttt{manifest\_in}(体现于)、\texttt{requires\_spatial\_config}(要求空间配置)
    \item \textbf{行为关系:}\texttt{performed\_at}(在……举行)、\texttt{requires\_role}(要求角色)、\texttt{follows\_sequence}(遵循程序)
\end{itemize}

\paragraph{多源数据融合与知识抽取}

知识库的数据来源具有多源异构性,需要针对不同数据类型设计抽取流程:

\begin{itemize}[leftmargin=2em, label={}]
    \item \textbf{结构化数据处理:}
    \begin{enumerate}[label=\alph*., leftmargin=3em]
        \item 来源:测绘图纸(CAD)、建筑档案数据库
        \item 抽取内容:空间单元的几何尺寸、拓扑关系、方位信息
        \item 方法:脚本解析 + 人工校验,转化为图谱节点与关系
    \end{enumerate}
    
    \item \textbf{半结构化数据处理:}
    \begin{enumerate}[label=\alph*., leftmargin=3em]
        \item 来源:族谱(世系表、族规、祭祀条目)、方志(祠堂志、风俗志)
        \item 抽取内容:宗族角色、礼制规则、仪式程序、历史沿革
        \item 方法:大语言模型辅助实体关系抽取 → 人工校验 → 结构化入库
        \item 关键挑战:古文理解、隐含规则显性化、跨文献一致性
    \end{enumerate}
    
    \item \textbf{非结构化数据处理:}
    \begin{enumerate}[label=\alph*., leftmargin=3em]
        \item 来源:建筑图像、仪式影像、口述访谈记录
        \item 抽取内容:空间元素实体(来自图像)、行为描述(来自影像/访谈)
        \item 方法:语义分割 → 实体识别 → 图谱节点(与形式驱动分析框架衔接)
    \end{enumerate}
    
    \item \textbf{知识对齐与融合:}
    \begin{enumerate}[label=\alph*., leftmargin=3em]
        \item 跨源实体消歧:同一空间单元在测绘图、族谱、图像中的不同表述对齐
        \item 规则知识整合:《朱子家礼》通则与地方族规的层次化组织
        \item 时间维度标注:区分不同历史时期的规则变迁
    \end{enumerate}
\end{itemize}

% 知识抽取流程图
\begin{figure}[h]
\centering
\begin{tikzpicture}[node distance=2.5cm]
    % 数据源
    \node (cad) [data] {CAD图纸};
    \node (doc) [data, right=1cm of cad] {族谱方志};
    \node (img) [data, right=1cm of doc] {图像影像};
    
    % 处理层
    \node (parse) [process, below=1.5cm of cad] {脚本解析};
    \node (llm) [process, below=1.5cm of doc] {LLM抽取};
    \node (seg) [process, below=1.5cm of img] {语义分割};
    
    % 校验层
    \node (verify) [process, below=2cm of llm] {人工校验\\实体消歧};
    
    % 知识库
    \node (kg) [startstop, below=1.5cm of verify] {知识图谱};
    
    % 连接
    \draw [arrow] (cad) -- (parse);
    \draw [arrow] (doc) -- (llm);
    \draw [arrow] (img) -- (seg);
    
    \draw [arrow] (parse) -- (verify);
    \draw [arrow] (llm) -- (verify);
    \draw [arrow] (seg) -- (verify);
    
    \draw [arrow] (verify) -- (kg);
\end{tikzpicture}
\caption{多源数据融合与知识抽取流程}
\end{figure}

\paragraph{知识库实现与质量保障}

\begin{itemize}[leftmargin=2em, label={}]
    \item \textbf{技术架构:}
    \begin{enumerate}[label=\alph*., leftmargin=3em]
        \item 存储层:Neo4j图数据库(高效图查询) + RDF三元组存储(语义互操作)
        \item 查询层:SPARQL端点 + Cypher查询API + 自然语言查询接口(对接RAG)
        \item 可视化层:知识图谱浏览界面,支持实体关系探索与子图查询
    \end{enumerate}
    
    \item \textbf{质量验证机制:}
    \begin{enumerate}[label=\alph*., leftmargin=3em]
        \item 本体一致性检查:OWL推理器检测逻辑矛盾
        \item 领域专家评审:邀请建筑史、礼制研究专家审核关键知识条目
        \item 推理一致性测试:设计测试查询,验证推理结果的合理性
        \item 覆盖度评估:对照文献原文,评估知识抽取的完整性
    \end{enumerate}
\end{itemize}

% 技术架构图
\begin{figure}[h]
\centering
\begin{tikzpicture}[node distance=2cm]
    % 存储层
    \node (neo4j) [process] {Neo4j};
    \node (rdf) [process, right=1cm of neo4j] {RDF Store};
    
    % 查询层
    \node (sparql) [process, above=1.5cm of neo4j, xshift=-0.5cm] {SPARQL};
    \node (cypher) [process, above=1.5cm of neo4j, xshift=0.5cm] {Cypher};
    \node (nl) [process, above=1.5cm of rdf] {NL接口};
    
    % 可视化层
    \node (viz) [startstop, above=1.5cm of cypher] {可视化界面};
    
    % 应用层
    \node (app) [startstop, above=1.5cm of viz] {分析应用\\(RAG/语义分割)};
    
    % 连接
    \draw [arrow] (neo4j) -- (sparql);
    \draw [arrow] (neo4j) -- (cypher);
    \draw [arrow] (rdf) -- (nl);
    
    \draw [arrow] (sparql) -- (viz);
    \draw [arrow] (cypher) -- (viz);
    \draw [arrow] (nl) -- (viz);
    
    \draw [arrow] (viz) -- (app);
    
    % 分组
    \begin{scope}[on background layer]
        \node[draw, dashed, fit=(neo4j)(rdf), label=below:存储层, inner sep=0.3cm] {};
        \node[draw, dashed, fit=(sparql)(cypher)(nl), label=below:查询层, inner sep=0.3cm] {};
    \end{scope}
\end{tikzpicture}
\caption{知识库技术架构}
\end{figure}

\subsubsection{形式驱动的人工智能分析框架:视觉识别与空间模式提取}

本部分对应符号学分析的"符号识别"环节,旨在从视觉数据中自动提取广府宗祠的形式特征,为规则分析提供实证基础。

\paragraph{基于符号学媒介属性的语义分割体系设计}

传统建筑语义分割多按构件功能分类(梁、柱、墙、顶),本研究从符号学视角重新设计分割体系,按媒介属性区分"承载何种类型意义的元素":

\begin{itemize}[leftmargin=2em, label={}]
    \item \textbf{文本类元素}(语言符号):
    \begin{enumerate}[label=\alph*., leftmargin=3em]
        \item 匾额:堂号、祠名
        \item 楹联:教化训诫、宗族理念
        \item 神主牌:世系信息、昭穆之序
        \item 碑刻:族规、捐资、修缮记录
    \end{enumerate}
    
    \item \textbf{图像类元素}(图像符号):
    \begin{enumerate}[label=\alph*., leftmargin=3em]
        \item 木雕:人物故事、吉祥图案
        \item 石雕:门枕石、柱础纹饰
        \item 砖雕:墀头、照壁装饰
        \item 彩画:梁架彩绘、壁画
    \end{enumerate}
    
    \item \textbf{结构类元素}(空间符号):
    \begin{enumerate}[label=\alph*., leftmargin=3em]
        \item 梁架系统:抬梁、穿斗、瓜柱
        \item 屋顶形式:硬山、悬山、歇山
        \item 墙体界面:山墙、屏门、隔断
        \item 地面铺装:阶陛、甬道
    \end{enumerate}
\end{itemize}

这一分类的符号学意义在于:文本类元素直接承载语言意义("指示符号"),图像类元素通过图像隐喻承载意义("像似符号"),结构类元素通过空间关系承载意义("规约符号")。三类元素在意义生成中的作用机制不同。

% 符号学分类流程图
\begin{figure}[h]
\centering
\begin{tikzpicture}[node distance=2.5cm]
    \node (input) [startstop] {建筑图像};
    \node (seg) [process, below of=input] {语义分割};
    
    \node (text) [process, below left=2cm and 1cm of seg] {文本类\\(指示符号)};
    \node (image) [process, below of=seg] {图像类\\(像似符号)};
    \node (struct) [process, below right=2cm and 1cm of seg] {结构类\\(规约符号)};
    
    \node (meaning1) [data, below of=text] {语言意义};
    \node (meaning2) [data, below of=image] {隐喻意义};
    \node (meaning3) [data, below of=struct] {空间意义};
    
    \draw [arrow] (input) -- (seg);
    \draw [arrow] (seg) -- (text);
    \draw [arrow] (seg) -- (image);
    \draw [arrow] (seg) -- (struct);
    
    \draw [arrow] (text) -- (meaning1);
    \draw [arrow] (image) -- (meaning2);
    \draw [arrow] (struct) -- (meaning3);
\end{tikzpicture}
\caption{基于符号学媒介属性的分割体系}
\end{figure}

\paragraph{语义分割模型的训练与优化}

\begin{itemize}[leftmargin=2em]
    \item \textbf{基础架构选择:}基于Transformer的分割模型(如SegFormer、Mask2Former),利用预训练权重进行迁移学习
    \item \textbf{数据标注规范:}制定详细的标注指南,确保跨案例一致性;采用多人标注 + 仲裁机制控制质量
    \item \textbf{训练策略:}
    \begin{enumerate}[label=\alph*., leftmargin=3em]
        \item 数据增强:几何变换、色彩抖动、遮挡模拟
        \item 类别不平衡处理:文本类元素面积占比小但语义权重高,需采用加权损失函数
        \item 跨案例泛化:训练集-验证集按地域、时期分层抽样
    \end{enumerate}
    \item \textbf{评估指标:}mIoU(平均交并比)、各类别IoU、边界精度
\end{itemize}

\paragraph{空间模式提取与"文本中心"现象的量化验证}

语义分割结果服务于以下分析任务:

\begin{itemize}[leftmargin=2em, label={}]
    \item \textbf{元素分布规律分析:}
    \begin{enumerate}[label=\alph*., leftmargin=3em]
        \item 各类元素的面积占比统计
        \item 空间分布规律:轴线上集中度、高度分布、前后进分布
        \item 跨案例对比:不同规模、不同时期祠堂的分布差异
    \end{enumerate}
    
    \item \textbf{"文本中心"现象的量化验证:}本研究提出假说:广府宗祠的类型辨识度高度依赖文本类元素,移除文本后,祠堂在视觉上将难以与普通民居区分。验证方法包括:
    \begin{enumerate}[label=\alph*., leftmargin=3em]
        \item 物理占比 vs 语义权重:文本类元素面积占比小(<5\%),但携带关键类型信息
        \item 消融实验:训练两个分类器(区分祠堂/民居),比较"含文本"与"去文本"图像的分类准确率差异
        \item 符号学解释:文本作为"指示符号"直接锚定建筑身份,结构类元素作为"规约符号"需要规则知识才能解码
    \end{enumerate}
\end{itemize}

\paragraph{空间句法的"伦理句法"拓展}

在形式分析层面引入空间句法,并尝试"身份维度"的拓展:

\begin{itemize}[leftmargin=2em, label={}]
    \item \textbf{传统空间句法的应用:}
    \begin{enumerate}[label=\alph*., leftmargin=3em]
        \item 轴线分析:提取空间轴线系统,计算整合度、深度值
        \item 凸空间分析:划分凸空间单元,构建可达性图
        \item 可视性分析:视域图构建,分析视觉控制点
    \end{enumerate}
    
    \item \textbf{伦理句法的初步探索:}
    \begin{enumerate}[label=\alph*., leftmargin=3em]
        \item 核心思想:传统空间句法计算"任何人"的可达性,伦理句法计算"特定身份"的许可性
        \item 实现方式:为空间句法图叠加"规则约束层"
        \begin{itemize}
            \item 定义身份类型:\{长房嫡长、庶出、女眷、外姓、宾客……\}
            \item 定义规则约束:\{寝堂-限制-女眷常入, 正堂-限制-外姓独入……\}
            \item 计算差异化可达性图谱:不同身份"看到"的空间拓扑结构不同
        \end{itemize}
        \item 预期产出:可视化展示不同身份的"可达空间"差异,为规则驱动分析提供形式基础
    \end{enumerate}
\end{itemize}

% 伦理句法概念图
\begin{figure}[h]
\centering
\begin{tikzpicture}[node distance=2.5cm]
    \node (space) [startstop] {空间拓扑};
    
    \node (trad) [process, below left=2cm and 2cm of space] {传统空间句法\\(无身份)};
    \node (ethical) [process, below right=2cm and 2cm of space] {伦理句法\\(有身份)};
    
    \node (anyone) [data, below of=trad] {任何人可达性};
    \node (role) [data, below of=ethical] {特定身份许可性};
    
    \node (rule) [process, right=1.5cm of ethical] {礼制规则约束};
    
    \draw [arrow] (space) -- (trad);
    \draw [arrow] (space) -- (ethical);
    \draw [arrow] (trad) -- (anyone);
    \draw [arrow] (ethical) -- (role);
    \draw [arrow] (rule) -- (ethical);
\end{tikzpicture}
\caption{空间句法的"伦理句法"拓展}
\end{figure}

\subsubsection{规则驱动的人工智能分析框架:礼制逻辑表征与知识发现}

本部分对应符号学分析的"意义解码"环节,旨在揭示形式背后的规则生成逻辑,验证"规则驱动型建筑"理论的解释力。

\paragraph{礼制知识的系统抽取与形式化}

从文献中抽取礼制规则,并形式化为可计算的知识表达:

\begin{itemize}[leftmargin=2em, label={}]
    \item \textbf{文献来源的层次化:}
    \begin{enumerate}[label=\alph*., leftmargin=3em]
        \item 规范性文本(通则):《朱子家礼》——祠堂制度、祭祀程序、昭穆之序
        \item 地方性文本(变体):广府地区族谱、族规——地方化适应与变通
        \item 实践性文本(实例):个案祠堂的祭祀记录、修缮碑文——规则的具体化
    \end{enumerate}
\end{itemize}

\begin{itemize}[leftmargin=2em, label={}]
    \item \textbf{抽取对象的类型化:}
\end{itemize}

\begin{table}[h]
\centering
\caption{礼制规则类型与空间编码}
\small
\begin{tabular}{p{2cm}p{3.5cm}p{2.5cm}p{4.5cm}}
\toprule
\textbf{规则类型} & \textbf{内容示例} & \textbf{空间编码方式} & \textbf{图谱表达示例} \\
\midrule
等级规则 & 昭穆之序、嫡庶之分 & 轴线位置、左右次序 & 〈长房, has\_rank, 1〉→〈神主, locate\_at, 中龛〉 \\
禁忌规则 & 男女之别、内外之分 & 边界设置、出入限制 & 〈女眷, forbidden\_from, 寝堂常入〉→〈屏门, separate, 前堂/寝堂〉 \\
仪式规则 & 祭祀程序、站位次序 & 动线编排、空间序列 & 〈春祭, has\_sequence, [迎神→初献→亚献→终献→辞神]〉→〈动线, connect, [门→庭→堂→寝]〉 \\
营建规则 & 规模限制、形制规定 & 开间数、进深、高度 & 〈三品以下, restricted\_to, 三间〉→〈祠堂, has\_bay, 3〉 \\
\bottomrule
\end{tabular}
\end{table}

\begin{itemize}[leftmargin=2em, label={}]
    \item \textbf{形式化表达方法:}
    \begin{enumerate}[label=\alph*., leftmargin=3em]
        \item 采用OWL本体语言定义规则类与关系
        \item 利用SWRL规则表达条件性规则(如"若为嫡长子,则神主位于中龛")
        \item 支持规则的继承与覆盖(地方规则覆盖通则的机制)
    \end{enumerate}
\end{itemize}

\paragraph{基于RAG的语义推理系统构建}

构建检索增强生成(Retrieval-Augmented Generation)系统,实现基于知识图谱的语义推理。

\begin{itemize}[leftmargin=2em, label={}]
    \item \textbf{系统架构:}
\end{itemize}

\begin{verbatim}
用户查询 → 意图识别 → 图谱检索(子图提取) → 上下文构建 
         → LLM推理生成 → 推理路径溯源 → 结果呈现
\end{verbatim}

% RAG系统架构图
\begin{figure}[h]
\centering
\begin{tikzpicture}[node distance=1.8cm]
    \node (query) [startstop] {用户查询};
    \node (intent) [process, below of=query] {意图识别};
    \node (retrieve) [process, below of=intent] {图谱检索\\(子图提取)};
    \node (context) [process, below of=retrieve] {上下文构建};
    \node (llm) [process, below of=context] {LLM推理生成};
    \node (trace) [process, below of=llm] {推理路径溯源};
    \node (result) [startstop, below of=trace] {结果呈现};
    
    \node (kg) [data, right=2cm of retrieve] {知识图谱};
    \node (doc) [data, right=2cm of context] {原始文献};
    
    \draw [arrow] (query) -- (intent);
    \draw [arrow] (intent) -- (retrieve);
    \draw [arrow] (retrieve) -- (context);
    \draw [arrow] (context) -- (llm);
    \draw [arrow] (llm) -- (trace);
    \draw [arrow] (trace) -- (result);
    
    \draw [arrow] (kg) -- (retrieve);
    \draw [arrow] (doc) -- (context);
\end{tikzpicture}
\caption{基于RAG的语义推理系统架构}
\end{figure}

\begin{itemize}[leftmargin=2em, label={}]
    \item \textbf{核心功能模块:}
    \begin{enumerate}[label=\alph*., leftmargin=3em]
        \item 图谱检索器:基于查询意图提取相关子图,提供推理所需的结构化知识
        \item 上下文构建器:将子图信息、原始文献片段、案例数据组织为LLM可理解的上下文
        \item 推理生成器:大语言模型基于上下文进行推理,生成答案
        \item 溯源模块:标注答案中每个断言的知识来源(图谱节点/文献出处)
    \end{enumerate}
\end{itemize}

\begin{itemize}[leftmargin=2em, label={}]
    \item \textbf{推理任务类型设计:}
\end{itemize}

\begin{table}[h]
\centering
\caption{RAG推理任务类型}
\begin{tabular}{p{3cm}p{5.5cm}p{4cm}}
\toprule
\textbf{任务类型} & \textbf{示例问题} & \textbf{推理要求} \\
\midrule
事实查询 & "陈家祠正堂供奉几代祖先?" & 检索图谱,返回事实 \\
规则推理 & "根据昭穆制度,三世祖应位于何位?" & 应用礼制规则进行推演 \\
情境判断 & "春祭时,庶出子孙应站于何处?" & 结合身份、仪式、规则综合判断 \\
反事实推理 & "若某人为外姓入赘,其祭祀权限如何?" & 假设情境下的规则推演 \\
跨案例比较 & "A祠与B祠在女性空间配置上有何异同?" & 跨案例检索与对比分析 \\
\bottomrule
\end{tabular}
\end{table}

\paragraph{知识发现与核心命题验证}

通过系统分析产出知识发现,并验证理论核心命题:

\begin{itemize}[leftmargin=2em, label={}]
    \item \textbf{核心发现一:转译结构的稳定性验证}
    \begin{enumerate}[label=\alph*., leftmargin=3em]
        \item 分析方法:跨案例对比同一规则类型的空间编码方式
        \item 预期结果:尽管不同祠堂的具体形式有差异,但"等级→轴线""禁忌→边界""仪式→动线"的转译模式高度一致
        \item 理论意义:实证支持"规则驱动型建筑"理论——类型的持久性在于转译结构层
    \end{enumerate}
    
    \item \textbf{核心发现二:伦理句法的有效性验证}
    \begin{enumerate}[label=\alph*., leftmargin=3em]
        \item 分析方法:基于规则进行"身份-空间-行为"的推演,与空间句法分析结果对照
        \item 预期结果:伦理句法能够解释传统空间句法无法解释的现象(如物理可达但规则禁止的空间)
        \item 理论意义:验证空间句法"身份维度"拓展的可行性与解释力
    \end{enumerate}
    
    \item \textbf{核心发现三:典型案例的深度分析}
    \begin{enumerate}[label=\alph*., leftmargin=3em]
        \item 入口选择的规则推理:为何祠堂多从侧门出入而非正门?(等级规则的空间编码)
        \item 祭祀站位的等级推演:昭穆制度如何转化为具体的站位安排?(仪式规则的空间实现)
        \item 性别空间的禁忌边界:女性空间的限制如何体现在空间配置中?(禁忌规则的边界设置)
    \end{enumerate}
\end{itemize}

\paragraph{框架整合与理论回应}

将形式驱动分析与规则驱动分析整合,回应理论层面的核心问题:

\begin{itemize}[leftmargin=2em, label={}]
    \item \textbf{对罗西类型学拓展的实证支持:}
    \begin{enumerate}[label=\alph*., leftmargin=3em]
        \item 形式分析表明广府宗祠形式不具独特性(与民居趋同)
        \item 规则分析表明转译结构具有稳定性(跨案例一致)
        \item 结论:验证"转译结构持久性"命题,支持"规则驱动型建筑"理论
    \end{enumerate}
    
    \item \textbf{对"记忆装置"操作化的有效性验证:}
    \begin{enumerate}[label=\alph*., leftmargin=3em]
        \item 通过知识图谱形式化表达"编码装置"的运作机制
        \item 通过RAG推理验证"解码路径"的可行性
        \item 结论:"规则的编码装置"概念具有可操作性
    \end{enumerate}
    
    \item \textbf{对方法体系的综合评估:}
    \begin{enumerate}[label=\alph*., leftmargin=3em]
        \item 符号学视角的解释力:成功弥合"形式描述-意义解释"之间的鸿沟
        \item 空间句法拓展的可行性:"伦理句法"概念的初步验证
        \item 多模态AI整合的效果:视觉-符号跨模态分析路径的有效性
    \end{enumerate}
\end{itemize}

\subsubsection{各研究内容之间的逻辑关系}

上述四部分研究内容构成紧密关联的整体:

\begin{figure}[h]
\centering
\begin{tikzpicture}[node distance=2.5cm, every node/.style={font=\small}]
    % 理论层
    \node (theory) [process, minimum width=8cm, minimum height=1.2cm] {4.1.1 理论建构\\(规则驱动型建筑理论 + 配套方法的理论整合)};
    
    % 数据层
    \node (data) [process, below=1.5cm of theory, minimum width=8cm, minimum height=1.2cm] {4.1.2 知识库构建\\(CIDOC CRM本体 + 多源知识抽取)};
    
    % 分析层
    \node (form) [process, below left=2cm and 1cm of data, minimum width=3.5cm] {4.1.3 形式驱动分析\\(语义分割+空间句法)};
    \node (rule) [process, below right=2cm and 1cm of data, minimum width=3.5cm] {4.1.4 规则驱动分析\\(礼制抽取+RAG推理)};
    
    % 描述
    \node (what) [below=0.3cm of form, font=\footnotesize] {发现现象(What)};
    \node (why) [below=0.3cm of rule, font=\footnotesize] {解释机制(Why)};
    
    \node (form2) [below=0.8cm of what, font=\footnotesize] {形式层特征提取};
    \node (rule2) [below=0.8cm of why, font=\footnotesize] {规则层逻辑还原};
    
    % 整合层
    \node (integrate) [process, below=3.5cm of data, minimum width=8cm] {理论命题验证 + 知识发现 + 方法评估};
    
    % 连接
    \draw [arrow] (theory) -- node[right, font=\footnotesize] {提供理论框架} (data);
    \draw [arrow] (data) -- node[left, font=\footnotesize, xshift=-0.5cm] {提供数据基础} (form);
    \draw [arrow] (data) -- node[right, font=\footnotesize, xshift=0.5cm] {提供数据基础} (rule);
    \draw [arrow] (form) -- (integrate);
    \draw [arrow] (rule) -- (integrate);
    
    % 双向箭头
    \draw [bidiarrow] (form) -- node[above, font=\footnotesize] {相互支撑} (rule);
\end{tikzpicture}
\caption{研究内容逻辑关系图}
\end{figure}

\begin{itemize}[leftmargin=2em]
    \item \textbf{4.1.1为全文提供理论框架:}规则驱动型建筑理论定义了研究问题,配套方法的理论整合指引了技术路径
    \item \textbf{4.1.2为后续分析提供数据基础:}知识库是形式分析与规则分析共享的基础设施
    \item \textbf{4.1.3与4.1.4构成双轨分析:}形式驱动分析发现现象,规则驱动分析提供解释,二者相互支撑
    \item \textbf{最终整合实现理论验证:}通过案例分析检验理论命题,评估方法有效性
\end{itemize}














% ---------------------------------------------------------------------
% 4.2 实施方案及其可行性论证
% ---------------------------------------------------------------------
\subsection{实施方案及其可行性论证}

\subsubsection{总体技术路线}

本研究采用"理论引领—方法支撑—实证验证"的总体技术路线,形成从理论建构到知识库构建、从形式分析到规则解析、最终整合验证的完整研究闭环。

\begin{figure}[htbp]
\centering
\begin{tikzpicture}[
    node distance=1.5cm,
    box/.style={rectangle, draw, fill=blue!10, text width=3.5cm, align=center, minimum height=1cm, rounded corners},
    arrow/.style={->, >=stealth, thick},
    stage/.style={rectangle, draw, fill=gray!20, text width=14cm, align=left, minimum height=3cm}
]

% 第一阶段
\node[stage] (stage1) at (0,0) {
    \textbf{第一阶段:理论建构}\\[0.3cm]
    罗西类型学文献研究 $\rightarrow$ 批判性分析适用边界界定 $\rightarrow$ "规则驱动型建筑"理论概念界定与命题提出\\[0.2cm]
    $\downarrow$\\
    配套方法体系的理论整合(符号学+空间句法+知识工程)
};

% 第二阶段
\node[stage, below=of stage1] (stage2) {
    \textbf{第二阶段:知识库构建}\\[0.3cm]
    CIDOC CRM框架适配 $\rightarrow$ 领域本体设计与实现 $\rightarrow$ 多源数据融合与知识抽取\\[0.2cm]
    $\downarrow$\\
    岭南传统建筑知识库(Neo4j + RDF)
};

% 第三阶段
\node[stage, below=of stage2] (stage3) {
    \textbf{第三阶段:双轨分析框架}\\[0.3cm]
    形式驱动分析框架(语义分割$\rightarrow$空间模式$\rightarrow$伦理句法)\\
    规则驱动分析框架(礼制知识抽取$\rightarrow$转译结构形式化$\rightarrow$RAG推理系统)\\[0.2cm]
    $\downarrow$\\
    跨模态整合分析
};

% 第四阶段
\node[stage, below=of stage3] (stage4) {
    \textbf{第四阶段:实证验证与理论回应}\\[0.3cm]
    广府宗祠案例分析 $\rightarrow$ 知识发现与模式归纳 $\rightarrow$ 理论命题验证、方法有效性评估
};

\draw[arrow] (stage1) -- (stage2);
\draw[arrow] (stage2) -- (stage3);
\draw[arrow] (stage3) -- (stage4);

\end{tikzpicture}
\caption{总体技术路线图}
\end{figure}

\subsubsection{各研究内容的具体实施方案}

\paragraph{理论建构的实施方案}

\textbf{1. 文献研究方案}

\begin{table}[htbp]
\centering
\caption{文献研究方案}
\begin{tabular}{llll}
\toprule
文献类型 & 核心文献 & 研究任务 & 产出形式 \\
\midrule
类型学理论 & Rossi《城市建筑》、Muratori学派著作、 & 梳理核心概念演化,识别理论 & 类型学理论谱系图、 \\
 & Caniggia类型过程理论 & 预设与适用边界 & 概念辨析表 \\
\midrule
空间分析方法 & Hillier空间句法系列、Stiny形状语法 & 分析方法论特征,识别拓展空间 & 方法论比较矩阵 \\
\midrule
符号学理论 & Peirce符号学、Eco建筑符号学 & 提炼适用于建筑分析的概念工具 & 符号学分析框架 \\
\midrule
华南宗族研究 & Freedman、科大卫、《朱子家礼》研究 & 理解礼制-空间关系的人类学背景 & 礼制规则类型表 \\
\bottomrule
\end{tabular}
\end{table}

\textbf{2. 概念建构方案}

"规则驱动型建筑"概念的建构遵循以下步骤:

\begin{itemize}
\item \textbf{步骤一:问题识别}
  \begin{itemize}
  \item 系统比对罗西类型学核心命题与广府宗祠经验
  \item 明确理论空白:形式不独特的建筑如何具有类型身份?
  \end{itemize}

\item \textbf{步骤二:概念界定}
  \begin{itemize}
  \item 采用"属+种差"定义法:规则驱动型建筑是一类(属)形态生成主要受社会规则支配的(种差)建筑类型
  \item 建立与相关概念的边界:与"形式驱动型建筑""功能驱动型建筑"的区分
  \end{itemize}

\item \textbf{步骤三:命题推演}
  \begin{itemize}
  \item 核心命题:"对于规则驱动型建筑,类型的持久性在于转译结构层而非形式层"
  \item 推导子命题:关于"记忆装置"操作化、"伦理句法"等
  \end{itemize}

\item \textbf{步骤四:理论整合}
  \begin{itemize}
  \item 将符号学、空间句法、知识工程纳入统一的理论框架
  \item 阐明各方法的理论定位与相互关系
  \end{itemize}
\end{itemize}

\textbf{3. 产出与验证}

\begin{table}[htbp]
\centering
\caption{产出与验证}
\begin{tabular}{ll}
\toprule
产出物 & 验证方式 \\
\midrule
"规则驱动型建筑"概念界定 & 理论自洽性检验、与现有理论的兼容性分析 \\
对罗西概念的补充命题 & 逻辑推演检验、后续实证验证 \\
配套方法的理论整合框架 & 内部一致性检验、可操作性评估 \\
\bottomrule
\end{tabular}
\end{table}

\paragraph{知识库构建的实施方案}

\textbf{1. 领域本体设计方案}

\textbf{(a) 本体开发方法论}

采用"自顶向下+自底向上"相结合的本体开发方法:

\begin{itemize}
\item \textbf{自顶向下}:从CIDOC CRM核心类出发,通过子类化(subClassOf)扩展领域类
\item \textbf{自底向上}:从实际案例数据中归纳实体类型与关系模式,反馈本体设计
\end{itemize}

\textbf{(b) 本体设计流程}

\begin{figure}[htbp]
\centering
\begin{tikzpicture}[
    node distance=2cm,
    box/.style={rectangle, draw, fill=blue!10, text width=4cm, align=center, minimum height=1cm, rounded corners},
    arrow/.style={->, >=stealth, thick}
]

\node[box] (req) {需求分析\\明确本体需服务的查询与推理任务};
\node[box, below=of req] (extract) {概念抽取\\从文献、案例中识别核心概念};
\node[box, below=of extract] (map) {CIDOC CRM映射\\建立领域类与CIDOC CRM的对应关系};
\node[box, below=of map] (rel) {关系定义\\定义领域特有的属性与关系};
\node[box, below=of rel] (constraint) {约束规则\\定义SWRL规则表达条件性知识};
\node[box, below=of constraint] (iter) {迭代优化\\通过实例化测试发现问题并修订};

\draw[arrow] (req) -- (extract);
\draw[arrow] (extract) -- (map);
\draw[arrow] (map) -- (rel);
\draw[arrow] (rel) -- (constraint);
\draw[arrow] (constraint) -- (iter);

\end{tikzpicture}
\caption{本体设计流程}
\end{figure}

\textbf{(c) 核心本体结构}

\begin{figure}[htbp]
\centering
\begin{tikzpicture}[
    node distance=0.8cm and 2cm,
    % 样式定义
    cidoc/.style={rectangle, draw, fill=orange!20, text width=4cm, align=left, minimum height=0.8cm},
    domstyle/.style={rectangle, draw, fill=green!20, text width=4.5cm, align=left, minimum height=0.8cm},
    sub/.style={rectangle, draw, fill=green!10, text width=4cm, align=left, minimum height=0.6cm, font=\small},
    arrow/.style={->, >=stealth}
]

% CIDOC CRM classes (左侧一列)
\node[cidoc] (obj) {E22 Man-Made Object};
\node[cidoc, below=3cm of obj] (place) {E53 Place};
\node[cidoc, below=3cm of place] (activity) {E7 Activity};
\node[cidoc, below=2cm of activity] (person) {E21 Person / E74 Group};
\node[cidoc, below=2cm of person] (prop) {E89 Propositional Object};

% Domain classes - Building
\node[domstyle, right=of obj] (building) {LNA\_Building (建筑实体)};
\node[sub, below=0.3cm of building, xshift=-0.5cm] (hall) {LNA\_AncestralHall(宗祠)};
\node[sub, below=0.2cm of hall] (residence) {LNA\_Residence(民居)};
\node[sub, below=0.2cm of residence] (temple) {LNA\_Temple(庙宇)};

% 修复报错的关键行:使用 domstyle 代替 domain
\node[domstyle, below=0.5cm of temple] (element) {LNA\_ArchElement(建筑构件)};
\node[sub, below=0.3cm of element, xshift=-0.5cm] (text) {LNA\_TextElement(文本类)};
\node[sub, below=0.2cm of text, xshift=0.3cm, font=\footnotesize] (plaque) {Plaque(匾额), Couplet(楹联)};

% Domain classes - Place
\node[domstyle, right=of place] (spatial) {LNA\_SpatialUnit(空间单元)};
\node[sub, below=0.3cm of spatial, xshift=-0.5cm] (mainhall) {MainHall(正堂)};
\node[sub, below=0.2cm of mainhall] (courtyard) {Courtyard(天井)};

% Domain classes - Activity
\node[domstyle, right=of activity] (ritual) {LNA\_Ritual(仪式活动)};
\node[sub, below=0.3cm of ritual, xshift=-0.5cm] (sacrifice) {AncestralSacrifice(祭祖)};

% Domain classes - Role
\node[domstyle, right=of person] (role) {LNA\_ClanRole(宗族角色)};
\node[sub, below=0.3cm of role, xshift=-0.5cm] (elder) {ElderBranch(长房)};

% Domain classes - Rule
\node[domstyle, right=of prop] (rule) {LNA\_RitualRule(礼制规则)};
\node[sub, below=0.3cm of rule, xshift=-0.5cm] (hierarchy) {HierarchyRule(等级规则)};

% Arrows (连接线)
\draw[arrow] (obj) -- (building);
\draw[arrow] (obj) -- (element);
\draw[arrow] (place) -- (spatial);
\draw[arrow] (activity) -- (ritual);
\draw[arrow] (person) -- (role);
\draw[arrow] (prop) -- (rule);

\end{tikzpicture}
\caption{核心本体结构}
\end{figure}

\textbf{2. 知识抽取方案}

\textbf{(a) 礼制文献知识抽取流程}

\begin{figure}[htbp]
\centering
\begin{tikzpicture}[
    node distance=1.5cm,
    box/.style={rectangle, draw, fill=blue!10, text width=5cm, align=left, minimum height=1cm, rounded corners},
    descbox/.style={rectangle, draw=none, fill=gray!10, text width=6cm, align=left, font=\small},
    arrow/.style={->, >=stealth, thick}
]

\node[box] (pre) {文献预处理};
\node[descbox, right=0.2cm of pre] {古籍数字化文本清洗、断句、标点};

\node[box, below=of pre] (llm) {LLM辅助抽取\\
\footnotesize{Prompt设计:\\
- 任务定义:从文本中抽取礼制规则三元组\\
- 实体类型约束:\{角色,空间,行为,规则\}\\
- 关系类型约束:\{governs,restricts,prescribes\}\\
- 输出格式规范:JSON结构化输出\\
- Few-shot示例:提供3-5个标注样例}};

\node[box, below=of llm] (verify) {人工校验};
\node[descbox, right=0.2cm of verify] {领域专家审核,标注置信度};

\node[box, below=of verify] (disamb) {实体消歧};
\node[descbox, right=0.2cm of disamb] {跨文献同一实体的识别与合并};

\node[box, below=of disamb] (store) {图谱入库};
\node[descbox, right=0.2cm of store] {转化为Neo4j节点与关系};

\draw[arrow] (pre) -- (llm);
\draw[arrow] (llm) -- (verify);
\draw[arrow] (verify) -- (disamb);
\draw[arrow] (disamb) -- (store);

\end{tikzpicture}
\caption{礼制文献知识抽取流程}
\end{figure}

\textbf{(b) 视觉数据知识抽取流程}

\begin{figure}[htbp]
\centering
\begin{tikzpicture}[
    node distance=1.2cm and 1.5cm,
    box/.style={rectangle, draw, fill=blue!10, text width=2.5cm, align=center, minimum height=0.8cm, rounded corners},
    arrow/.style={->, >=stealth, thick}
]

\node[box] (img) {建筑图像};
\node[box, right=of img] (seg) {语义分割模型};
\node[box, right=of seg] (inst) {构件实例};
\node[box, right=of inst] (attr) {实体属性抽取};
\node[box, right=of attr] (node) {图谱节点};

\node[box, below=of inst] (spatial) {空间关系推断};
\node[box, below=of attr] (feature) {视觉特征向量};
\node[box, below=of spatial] (edge) {图谱关系};

\draw[arrow] (img) -- (seg);
\draw[arrow] (seg) -- (inst);
\draw[arrow] (inst) -- (attr);
\draw[arrow] (attr) -- (node);
\draw[arrow] (inst) -- (spatial);
\draw[arrow] (attr) -- (feature);
\draw[arrow] (spatial) -- (edge);
\draw[arrow] (feature) -- (edge);
\draw[arrow] (img) |- (edge);

\end{tikzpicture}
\caption{视觉数据知识抽取流程}
\end{figure}

\textbf{(c) 跨源知识对齐方案}

\begin{table}[htbp]
\centering
\caption{跨源知识对齐方案}
\begin{tabular}{lll}
\toprule
对齐任务 & 方法 & 验证机制 \\
\midrule
空间单元对齐 & 测绘图标注名称$\leftrightarrow$族谱空间描述$\leftrightarrow$图像语义分割结果 & 位置、形态特征匹配 \\
角色实体对齐 & 族谱世系表$\leftrightarrow$祭祀记录$\leftrightarrow$碑刻人名 & 姓名、世代、关系匹配 \\
规则知识对齐 & 《朱子家礼》通则$\leftrightarrow$地方族规$\leftrightarrow$实践记录 & 层次化组织,标注来源 \\
\bottomrule
\end{tabular}
\end{table}

\textbf{3. 知识库技术实现}

\textbf{(a) 存储架构}

\begin{figure}[htbp]
\centering
\begin{tikzpicture}[
    node distance=0.8cm,
    layer/.style={rectangle, draw, fill=gray!20, text width=12cm, align=center, minimum height=0.8cm, font=\bfseries},
    box/.style={rectangle, draw, fill=blue!10, text width=3.5cm, align=center, minimum height=0.7cm, font=\small},
    storage/.style={rectangle, draw, fill=green!10, text width=11cm, align=left, minimum height=1.2cm, font=\small}
]

\node[layer] (app) {应用层};
\node[box, below=0.3cm of app, xshift=-4cm] (browser) {知识浏览器};
\node[box, below=0.3cm of app] (rag) {RAG问答接口};
\node[box, below=0.3cm of app, xshift=4cm] (api) {分析工具API};

\node[layer, below=1.2cm of app] (query) {查询层};
\node[box, below=0.3cm of query, xshift=-4cm] (cypher) {Cypher查询};
\node[box, below=0.3cm of query] (sparql) {SPARQL端点};
\node[box, below=0.3cm of query, xshift=4cm] (vector) {向量检索};

\node[layer, below=1.2cm of query] (store) {存储层};
\node[storage, below=0.3cm of store] (neo4j) {
Neo4j图数据库\\
- 实体节点:空间、构件、角色、仪式、规则\\
- 关系边:空间关系、规则关系、转译关系\\
- 属性:几何参数、时间标注、来源标注
};
\node[storage, below=0.2cm of neo4j] (rdf) {
RDF三元组存储(Apache Jena)\\
- OWL本体定义 \quad - SWRL推理规则 \quad - 语义互操作接口
};
\node[storage, below=0.2cm of rdf] (milvus) {
向量数据库(Milvus/Pinecone)\\
- 文本嵌入向量:文献片段、规则描述\\
- 视觉特征向量:构件图像特征
};

\end{tikzpicture}
\caption{知识库技术架构}
\end{figure}

\textbf{(b) 质量保障机制}

\begin{table}[htbp]
\centering
\caption{质量保障机制}
\begin{tabular}{llll}
\toprule
质量维度 & 检查方法 & 通过标准 \\
\midrule
本体一致性 & OWL推理器(HermiT)检测逻辑矛盾 & 无unsatisfiable类 \\
实例完整性 & SHACL约束验证必填属性 & 违规率<5\% \\
抽取准确性 & 抽样人工复核(10\%样本) & 准确率>85\% \\
推理合理性 & 设计测试查询集(50条) & 正确率>90\% \\
\bottomrule
\end{tabular}
\end{table}

\paragraph{形式驱动分析框架的实施方案}

\textbf{1. 语义分割模型开发方案}

\textbf{(a) 数据准备}

\begin{table}[htbp]
\centering
\caption{数据准备方案}
\begin{tabular}{llll}
\toprule
数据类型 & 来源 & 数量目标 & 标注规范 \\
\midrule
训练图像 & 田野调查拍摄 & 2000张以上 & 像素级标注,15类 \\
验证图像 & 分层抽样 & 400张 & 与训练集同标准 \\
测试图像 & 独立案例 & 200张 & 跨案例泛化测试 \\
\bottomrule
\end{tabular}
\end{table}

标注类别体系:
\begin{itemize}
\item 文本类(4类):匾额、楹联、神主牌、碑刻
\item 图像类(4类):木雕、石雕、砖雕、彩画
\item 结构类(6类):梁架、柱础、屋顶、墙体、门窗、地面
\item 背景类(1类):背景
\end{itemize}

\textbf{(b) 模型训练流程}

\begin{figure}[htbp]
\centering
\begin{tikzpicture}[
    node distance=1.5cm,
    box/.style={rectangle, draw, fill=blue!10, text width=4.5cm, align=left, minimum height=1cm, rounded corners},
    descbox/.style={rectangle, draw=none, fill=gray!10, text width=6cm, align=left, font=\small},
    arrow/.style={->, >=stealth, thick}
]

\node[box] (pretrain) {预训练模型选择与加载};
\node[descbox, right=0.2cm of pretrain] {SegFormer-B3 / Mask2Former\\(ImageNet + ADE20K预训练权重)};

\node[box, below=of pretrain] (aug) {数据增强策略配置};
\node[descbox, right=0.2cm of aug] {几何变换(旋转、翻转、裁剪)\\色彩抖动(亮度、对比度、饱和度)\\遮挡模拟(随机擦除)};

\node[box, below=of aug] (loss) {损失函数设计};
\node[descbox, right=0.2cm of loss] {加权交叉熵损失(解决类别不平衡)\\+ Dice Loss(优化边界)\\文本类权重:5x,结构类权重:1x};

\node[box, below=of loss] (train) {训练策略};
\node[descbox, right=0.2cm of train] {学习率:余弦退火,初始1e-4\\批量大小:8(梯度累积至32)\\训练轮次:100 epochs,早停patience=10};

\node[box, below=of train] (eval) {评估与调优};
\node[descbox, right=0.2cm of eval] {mIoU、各类别IoU、边界F1\\消融实验验证设计选择};

\draw[arrow] (pretrain) -- (aug);
\draw[arrow] (aug) -- (loss);
\draw[arrow] (loss) -- (train);
\draw[arrow] (train) -- (eval);

\end{tikzpicture}
\caption{语义分割模型训练流程}
\end{figure}

\textbf{(c) 评估指标与目标}

\begin{table}[htbp]
\centering
\caption{评估指标与目标}
\begin{tabular}{lll}
\toprule
指标 & 计算方式 & 目标值 \\
\midrule
mIoU & 所有类别IoU均值 & $\geq$65\% \\
文本类IoU & 匾额、楹联等4类均值 & $\geq$70\% \\
结构类IoU & 梁架、屋顶等6类均值 & $\geq$60\% \\
边界F1 & 边界像素的F1分数 & $\geq$50\% \\
\bottomrule
\end{tabular}
\end{table}

\textbf{2. 空间模式提取方案}

\textbf{(a) 元素分布分析流程}

\begin{figure}[htbp]
\centering
\begin{tikzpicture}[
    node distance=1.5cm,
    box/.style={rectangle, draw, fill=blue!10, text width=4.5cm, align=left, minimum height=1cm, rounded corners},
    descbox/.style={rectangle, draw=none, fill=gray!10, text width=5.5cm, align=left, font=\small},
    arrow/.style={->, >=stealth, thick}
]

\node[box] (input) {语义分割结果};

\node[box, below=of input] (area) {面积统计};
\node[descbox, right=0.2cm of area] {各类元素像素面积占比\\分别统计:整体/轴线区/边缘区};

\node[box, below=of area] (dist) {空间分布分析};
\node[descbox, right=0.2cm of dist] {轴线集中度:轴线区元素占比\\高度分布:各高度区间元素分布\\前后分布:各进深区间元素分布};

\node[box, below=of dist] (compare) {跨案例比较};
\node[descbox, right=0.2cm of compare] {不同规模祠堂的分布模式差异\\不同时期祠堂的分布模式演变};

\node[box, below=of compare] (pattern) {模式归纳};
\node[descbox, right=0.2cm of pattern] {提取稳定的分布规律\\识别异常案例与变体};

\draw[arrow] (input) -- (area);
\draw[arrow] (area) -- (dist);
\draw[arrow] (dist) -- (compare);
\draw[arrow] (compare) -- (pattern);

\end{tikzpicture}
\caption{元素分布分析流程}
\end{figure}

\textbf{(b) "文本中心"假说验证实验设计}

\begin{table}[htbp]
\centering
\caption{"文本中心"假说验证实验设计}
\begin{tabular}{lll}
\toprule
实验 & 方法 & 预期结果 \\
\midrule
物理占比分析 & 统计文本类元素面积占比 & <5\%(面积占比极小) \\
分类消融实验 & 训练祠堂/民居二分类器,比较含/去文本图像准确率 & 去文本后准确率显著下降(>15\%) \\
人工辨识测试 & 请非专业人员辨识去文本图像是否为祠堂 & 辨识准确率接近随机($\sim$50\%) \\
\bottomrule
\end{tabular}
\end{table}

\textbf{3. 伦理句法探索方案}

\textbf{(a) 传统空间句法分析}

\begin{table}[htbp]
\centering
\caption{传统空间句法分析}
\begin{tabular}{lll}
\toprule
分析类型 & 工具 & 产出 \\
\midrule
轴线分析 & depthmapX & 轴线图、整合度值、深度值 \\
凸空间分析 & 自开发脚本 & 凸空间划分、可达性矩阵 \\
可视性分析 & depthmapX & 视域图、视觉控制点识别 \\
\bottomrule
\end{tabular}
\end{table}

\textbf{(b) 伦理句法拓展实现}

\begin{figure}[htbp]
\centering
\begin{tikzpicture}[
    node distance=1.3cm,
    box/.style={rectangle, draw, fill=blue!10, text width=5cm, align=left, minimum height=1cm, rounded corners},
    descbox/.style={rectangle, draw=none, fill=gray!10, text width=5.5cm, align=left, font=\small},
    arrow/.style={->, >=stealth, thick}
]

\node[box] (physical) {1. 物理可达性图 $G_{physical}$};
\node[descbox, right=0.2cm of physical] {传统空间句法计算\\所有空间单元的拓扑连接关系};

\node[box, below=of physical] (rule) {2. 规则约束表 $R[role, space]$};
\node[descbox, right=0.2cm of rule] {从知识图谱提取\\$\{\langle$女眷,寝堂,禁入$\rangle$,\\$\langle$外姓,正堂,限入$\rangle$,...$\}$};

\node[box, below=of rule] (role) {3. 身份可达性图 $G_{role}$};
\node[descbox, right=0.2cm of role] {$G_{role} = G_{physical} - Blocked(R, role)$\\不同身份"看到"的不同拓扑结构};

\node[box, below=of role] (metric) {4. 差异化指标};
\node[descbox, right=0.2cm of metric] {各身份的整合度、深度值差异\\可达空间数量差异\\"控制空间"的识别};

\node[box, below=of metric] (vis) {5. 可视化输出};
\node[descbox, right=0.2cm of vis] {多层叠加图:不同身份的可达区域着色\\对比分析图:物理可达 vs 伦理许可};

\draw[arrow] (physical) -- (rule);
\draw[arrow] (rule) -- (role);
\draw[arrow] (role) -- (metric);
\draw[arrow] (metric) -- (vis);

\end{tikzpicture}
\caption{伦理句法计算流程}
\end{figure}

\paragraph{规则驱动分析框架的实施方案}

\textbf{1. 礼制知识抽取方案}

\textbf{(a) 文献处理层次}

\begin{table}[htbp]
\centering
\caption{文献处理层次}
\begin{tabular}{llll}
\toprule
层次 & 文献类型 & 处理方法 & 知识类型 \\
\midrule
通则层 & 《朱子家礼》 & 全文精读 + LLM辅助结构化 & 规范性规则 \\
地方层 & 广府族谱、族规 & 选择性精读 + LLM批量处理 & 地方变体 \\
实践层 & 祭祀记录、碑刻 & 案例导向选读 & 实例证据 \\
\bottomrule
\end{tabular}
\end{table}

\textbf{(b) 规则抽取Prompt模板}

\begin{verbatim}
【任务】从以下古文文本中抽取礼制规则,以JSON格式输出。

【实体类型】
- Role:宗族角色(如:长房、嫡子、庶出、女眷、外姓)
- Space:空间单元(如:正堂、寝堂、天井、两廊)
- Behavior:行为(如:祭祀、入内、站立、跪拜)
- Rule:规则描述

【关系类型】
- restricts_access:限制某角色进入/使用某空间
- prescribes_position:规定某角色在某空间的位置
- governs_sequence:规定仪式程序的顺序

【输出格式】
{
  "triples": [
    {"subject": {...}, "relation": "...", 
     "object": {...}, "source": "原文引用"}
  ]
}

【示例】
输入:"主人升自阼阶,西向"
输出:{
  "triples": [{
    "subject": {"type": "Role", "value": "主人"},
    "relation": "prescribes_position",
    "object": {"type": "Space", "value": "阼阶(东阶)"},
    "source": "主人升自阼阶,西向"
  }]
}

【待处理文本】
{input_text}
\end{verbatim}

\textbf{(c) 转译结构形式化}

将抽取的规则转化为图谱可表达的形式化结构:

\begin{table}[htbp]
\centering
\caption{转译结构形式化}
\begin{tabular}{lp{5cm}p{6cm}}
\toprule
规则类型 & 自然语言示例 & 形式化表达 \\
\midrule
等级规则 & "神主按昭穆之序排列" & (昭位祖先)-[rank:奇数代]->(左龛)\newline (穆位祖先)-[rank:偶数代]->(右龛) \\
\midrule
禁忌规则 & "妇人不入正寝" & (女眷)-[forbidden\_from]->(寝堂)\newline (屏门)-[separate]->(前堂,寝堂) \\
\midrule
仪式规则 & "祭礼序:迎神、初献、亚献、终献、辞神" & (春祭)-[has\_sequence]->(迎神$\rightarrow$初献$\rightarrow$亚献$\rightarrow$终献$\rightarrow$辞神)\newline (迎神)-[performed\_at]->(门)\newline (初献)-[performed\_at]->(正堂) \\
\bottomrule
\end{tabular}
\end{table}

\textbf{2. RAG语义推理系统实现方案}

\textbf{(a) 系统架构设计}

\begin{figure}[htbp]
\centering
\begin{tikzpicture}[
    node distance=1cm,
    module/.style={rectangle, draw, fill=blue!15, text width=11cm, align=left, minimum height=1.5cm, font=\small},
    subbox/.style={rectangle, draw, fill=green!10, text width=5cm, align=center, minimum height=0.7cm, font=\footnotesize},
    arrow/.style={->, >=stealth, thick}
]

\node[module] (query) {
\textbf{查询理解模块}\\
意图分类(5类推理) \quad 实体识别(角色/空间) \quad 查询改写(扩展同义词/上下位)
};

\node[module, below=of query] (retrieve) {
\textbf{知识检索模块}\\
图谱子图检索(Cypher模式匹配、多跳关系遍历) \quad 向量相似检索(文献片段语义匹配、案例相似度排序)
};

\node[module, below=of retrieve] (context) {
\textbf{上下文构建模块}\\
Context = \{query:用户原始问题, intent:识别的意图类型, subgraph:检索到的图谱子图(结构化),\\
\hspace{2.5cm}documents:相关文献片段(非结构化), examples:类似问答示例(Few-shot)\}
};

\node[module, below=of context] (reason) {
\textbf{推理生成模块}\\
LLM推理(GPT-4/Claude、结构化Prompt) \quad 溯源标注(答案$\rightarrow$图谱节点映射、答案$\rightarrow$文献出处标注)
};

\draw[arrow] (query) -- (retrieve);
\draw[arrow] (retrieve) -- (context);
\draw[arrow] (context) -- (reason);

\node[above=0.2cm of query] {用户查询};
\node[below=0.2cm of reason] {推理结果 + 溯源信息};

\end{tikzpicture}
\caption{RAG语义推理系统架构}
\end{figure}

\textbf{(b) 推理任务与评估设计}

\begin{table}[htbp]
\centering
\caption{推理任务与评估设计}
\begin{tabular}{llll}
\toprule
任务类型 & 示例问题 & 评估方法 & 目标指标 \\
\midrule
事实查询 & "陈家祠正堂供奉几代祖先?" & 与标注答案比对 & 准确率$\geq$95\% \\
规则推理 & "根据昭穆制度,三世祖应位于何位?" & 专家评审 & 正确率$\geq$85\% \\
情境判断 & "春祭时,庶出子孙可否进入寝堂?" & 专家评审+规则一致性 & 正确率$\geq$80\% \\
反事实推理 & "若无昭穆制度,神主排列会有何不同?" & 专家评审 & 合理性评分$\geq$4/5 \\
跨案例比较 & "A祠与B祠在女性空间配置上有何异同?" & 专家评审 & 完整性评分$\geq$4/5 \\
\bottomrule
\end{tabular}
\end{table}

\textbf{(c) 测试查询集设计}

构建包含100条测试查询的评估集:
\begin{itemize}
\item 事实查询:30条
\item 规则推理:25条
\item 情境判断:25条
\item 反事实推理:10条
\item 跨案例比较:10条
\end{itemize}

每条查询配备:标准答案(专家标注)、评分标准、溯源参考。

\textbf{3. 知识发现与案例分析方案}

\textbf{(a) 核心命题验证实验设计}

\begin{table}[htbp]
\centering
\caption{核心命题验证实验设计}
\begin{tabular}{lll}
\toprule
验证命题 & 实验方法 & 判定标准 \\
\midrule
转译结构的稳定性 & 跨10+案例比较同一规则类型的空间编码方式 & 一致性$\geq$80\%判定为稳定 \\
伦理句法的有效性 & 计算5+案例的身份差异化可达性,与规则知识对照 & 吻合度$\geq$85\%判定为有效 \\
"记忆装置"操作化 & 通过RAG推理还原形式背后的规则,专家评审还原准确性 & 准确性$\geq$80\%判定为成功 \\
\bottomrule
\end{tabular}
\end{table}

\textbf{(b) 典型案例深度分析方案}

选取3-5座典型广府宗祠进行深度分析,每个案例完成:

\begin{table}[htbp]
\centering
\caption{典型案例深度分析方案}
\begin{tabular}{lll}
\toprule
分析维度 & 具体内容 & 产出形式 \\
\midrule
形式分析 & 语义分割结果、空间模式提取、空间句法计算 & 分析报告+可视化图表 \\
规则分析 & 相关族谱/族规知识抽取、转译结构形式化 & 知识图谱子图+规则表 \\
整合分析 & 形式-规则对照、伦理句法计算、RAG推理验证 & 整合分析报告 \\
发现归纳 & 转译模式归纳、理论命题验证 & 发现总结表 \\
\bottomrule
\end{tabular}
\end{table}

\subsubsection{可行性论证}

\paragraph{理论可行性}

\textbf{1. 类型学理论基础的可靠性}

罗西类型学作为20世纪后半叶最具影响力的建筑理论之一,其核心概念(类型、持久性、记忆装置)已被建筑学界广泛接受。本研究不是否定罗西理论,而是在承认其核心洞见的基础上进行补充——针对其未充分关注的"形式模糊型"建筑提出拓展。这种"补充而非替代"的理论策略具有较高的可行性。

\textbf{2. 跨学科方法整合的理论依据}

\begin{itemize}
\item \textbf{符号学}:皮尔斯符号学的三元关系框架已被艾柯等学者成功应用于建筑分析,为本研究提供了理论先例。
\item \textbf{空间句法}:空间句法的学术史中已有多位学者尝试引入社会因素(如Hanson对住宅性别空间的研究),本研究的"身份维度"拓展是这一传统的延续。
\item \textbf{知识图谱}:CIDOC CRM作为文化遗产领域的国际本体标准(ISO 21127),为领域本体设计提供了可靠的上层框架。
\end{itemize}

\textbf{3. 华南宗族研究的学术积淀}

弗里德曼、科大卫等人类学家对华南宗族的深入研究,为理解广府宗祠的社会文化背景提供了坚实基础。《朱子家礼》研究领域的丰富成果为礼制规则抽取提供了学术参照。

\paragraph{技术可行性}

\textbf{1. 语义分割技术的成熟度}

\begin{table}[htbp]
\centering
\caption{语义分割技术的成熟度}
\begin{tabular}{llll}
\toprule
技术指标 & 当前水平 & 本研究需求 & 可行性评估 \\
\midrule
通用图像分割 & SOTA mIoU>85\% (ADE20K) & 领域适配 & 高度可行 \\
迁移学习能力 & 预训练模型丰富 & 小样本适配 & 高度可行 \\
细粒度分割 & 边界精度持续提升 & 构件级分割 & 基本可行 \\
\bottomrule
\end{tabular}
\end{table}

前期实验已在小规模数据集上验证了语义分割的可行性,达到mIoU 58\%,通过扩充数据和优化模型有望达到目标指标。

\textbf{2. 知识图谱技术的成熟度}

\begin{itemize}
\item \textbf{本体建模}:OWL/RDF技术栈成熟,工具链完善(Protégé、Jena等)
\item \textbf{知识抽取}:大语言模型(GPT-4、Claude)在结构化信息抽取任务上表现优异,配合人工校验可保障质量
\item \textbf{图谱存储}:Neo4j等图数据库技术成熟,查询性能可满足需求
\end{itemize}

前期已完成小规模知识图谱构建实验,验证了技术流程的可行性。

\textbf{3. RAG技术的可行性}

检索增强生成(RAG)技术近年发展迅速,已有多个成熟框架(LangChain、LlamaIndex)可供使用。本研究的创新点在于将RAG应用于建筑遗产知识推理这一特定领域,技术基础已具备,关键在于领域适配。

\textbf{4. 前期实验验证}

\begin{table}[htbp]
\centering
\caption{前期实验验证}
\begin{tabular}{lll}
\toprule
技术模块 & 前期实验情况 & 验证结论 \\
\midrule
语义分割 & 500张图像训练,mIoU 58\% & 技术可行,需扩充数据 \\
知识抽取 & 3篇族谱文献LLM抽取 & 技术可行,需设计校验流程 \\
图谱构建 & 100+节点试验图谱 & 技术可行,需完善本体 \\
RAG原型 & 20条查询测试 & 技术可行,需优化检索 \\
\bottomrule
\end{tabular}
\end{table}

\paragraph{数据可行性}

\textbf{1. 案例数据的可获取性}

\begin{table}[htbp]
\centering
\caption{案例数据的可获取性}
\begin{tabular}{llll}
\toprule
数据类型 & 来源 & 可获取性 & 数量预估 \\
\midrule
建筑图像 & 田野调查拍摄 & 高 & 3000+张 \\
测绘图纸 & 文物部门档案、公开出版物 & 中 & 20+座祠堂 \\
族谱文献 & 图书馆藏、数字化资源库 & 中 & 30+部 \\
方志文献 & 地方志数据库、公开出版物 & 高 & 10+部 \\
\bottomrule
\end{tabular}
\end{table}

\textbf{2. 礼制文献的数字化基础}

\begin{itemize}
\item 《朱子家礼》:已有多个数字化版本,配有注释
\item 广府族谱:部分已数字化,可通过图书馆获取
\item 地方志:多已收入《中国方志库》等数据库
\end{itemize}

\textbf{3. 田野调查的可执行性}

广府地区(广州、佛山、东莞、中山等)宗祠建筑保存相对完整,交通便利,田野调查可行性高。前期已完成10+座祠堂的初步调查,积累了调查经验。

\paragraph{时间可行性}

本研究计划在36个月内完成,各阶段时间分配如下:

\begin{table}[htbp]
\centering
\caption{时间可行性分析}
\begin{tabular}{llll}
\toprule
阶段 & 时间 & 核心任务 & 风险评估 \\
\midrule
理论建构 & 6个月 & 文献研究、概念建构 & 低风险 \\
知识库构建 & 6个月 & 本体设计、知识抽取 & 中风险(工作量大) \\
形式分析框架 & 6个月 & 模型训练、工具开发 & 中风险(技术挑战) \\
规则分析框架 & 6个月 & 规则抽取、RAG系统 & 中风险(技术挑战) \\
整合验证 & 6个月 & 案例分析、论文撰写 & 低风险 \\
缓冲预留 & 6个月 & 问题解决、论文修改 & — \\
\bottomrule
\end{tabular}
\end{table}

时间规划留有6个月缓冲期,可应对不可预见的困难。

\paragraph{资源可行性}

\textbf{1. 计算资源}

\begin{table}[htbp]
\centering
\caption{计算资源需求}
\begin{tabular}{lll}
\toprule
资源类型 & 需求 & 获取途径 \\
\midrule
GPU服务器 & 4$\times$A100用于模型训练 & 学校/云平台 \\
图数据库 & Neo4j企业版或社区版 & 开源/学术授权 \\
向量数据库 & Milvus/Pinecone & 开源/云服务 \\
LLM API & GPT-4/Claude调用额度 & 购买/学术计划 \\
\bottomrule
\end{tabular}
\end{table}

\textbf{2. 软件工具}

\begin{table}[htbp]
\centering
\caption{软件工具需求}
\begin{tabular}{lll}
\toprule
工具类型 & 具体工具 & 获取方式 \\
\midrule
深度学习框架 & PyTorch、Transformers & 开源 \\
本体编辑 & Protégé & 开源 \\
图数据库 & Neo4j & 开源/商业 \\
空间句法 & depthmapX & 开源 \\
RAG框架 & LangChain & 开源 \\
\bottomrule
\end{tabular}
\end{table}

\textbf{3. 人力资源}

\begin{itemize}
\item 研究者本人:具备深度学习、知识图谱、建筑学跨学科背景
\item 导师团队:可提供建筑遗产数字化、礼制研究的学术指导
\item 合作专家:可邀请礼制研究、古建筑专家参与校验
\end{itemize}


\textbf{综合评估}:本研究在理论、技术、数据、时间、资源等方面均具备可行性基础,核心风险可控,总体可行性评估为\textbf{可行}。