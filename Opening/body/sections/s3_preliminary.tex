\subsection{理论框架的前期研究}

\subsubsection{建筑空间模式表征的理论探索}

\paragraph{(1)建筑类型学的认识论困境:从西方经验到普遍问题}

前期研究首先对20世纪建筑类型学的核心文献进行了系统梳理,重点考察了从迪朗形式分类传统到罗西"集体记忆"理论的演进脉络。通过对罗西《城市建筑学》(\textit{The Architecture of the City})及穆拉托利学派相关文献的深入阅读,识别出经典类型学理论的一个核心特征:\textbf{类型的可识别性建立在形式的持久性与独特性基础之上}。

这一特征在西方纪念碑建筑传统中得到了充分验证——哥特教堂的尖拱、巴西利卡的中殿、帕拉第奥别墅的对称立面,均以鲜明的视觉特征承载类型身份。这为理解基于几何形式的建筑类型提供了有效的分析框架。

当研究视野拓展至非西方建筑传统时,前期研究通过对中国传统建筑文献的深入阅读,发现了一类具有不同表征逻辑的建筑现象:\textbf{形式呈现多样性,但类型身份保持稳定}。这类建筑的类型性主要体现为社会规则对空间配置的系统性支配,而非视觉形式的差异化特征。研究将这类建筑界定为"规则驱动型建筑",并提出其为类型学理论开拓了新的研究空间:当形式呈现可变性时,类型的持久性通过何种机制得以维系?这为拓展罗西"记忆装置"概念的适用范围提供了新的研究对象。

\paragraph{(2)"形式驱动"与"规则驱动":两类空间生成逻辑的比较}

为回应上述问题,前期研究尝试建立一个更具包容性的分析框架。通过对不同建筑类型生成机制的比较分析,初步提出"形式驱动型建筑"与"规则驱动型建筑"的概念区分。

所谓"形式驱动型建筑",是指类型身份主要通过视觉形式的独特性来承载的建筑。这类建筑的形式生成受风格演化、结构逻辑与美学追求的主导,类型的持久性体现在形式层面,分析路径可以直接从形式推导至意义。西方教堂、纪念碑建筑是这一类型的典型代表。

所谓"规则驱动型建筑",则是指类型身份主要通过社会规则对空间配置的系统性支配来承载的建筑。这类建筑的形式生成受礼制、禁忌、等级等社会规则的主导,类型的持久性不在形式层,而在规则与形式之间的转译结构层,分析路径需要从形式回溯规则,再从规则理解意义。中国宗族建筑是这一类型的典型代表。

这一区分并非绝对的二元对立,而是一个连续谱系的两端。多数建筑同时具有形式驱动与规则驱动的双重面向,但不同建筑类型在这一谱系上的位置存在显著差异。

前期研究通过对以下案例的初步考察,验证了这一分析框架的解释力。在中国宗祠建筑中,三间两廊的基本形制与普通民居高度趋同,视觉辨识度极低,但空间配置强烈受宗法礼制支配——正堂供奉祖先神主,天井区分内外,两廊安排次要功能,轴线体现尊卑等级。形式的模糊性与规则的严格性形成鲜明对照,使其成为规则驱动型建筑的典型。在西方哥特教堂中,尖拱、飞扶壁、玫瑰窗等视觉元素具有高度独特性且跨越数百年保持稳定,类型身份直接体现为形式特征,属于形式驱动型建筑的典型。

更值得注意的是,前期研究在考察明代社会礼制运作时发现,"规则驱动"的逻辑并不仅限于建筑领域,而是传统中国社会组织的普遍特征。明代礼制通过《大明集礼》《大明会典》等典籍,系统规定了从朝廷祭祀到民间冠婚丧祭的全部仪节,形成了一套覆盖社会各阶层的行为规范体系。这套规则体系具有高度的稳定性与传承性,而其具体的物质载体——服饰、器物、空间——则可以因时因地而有变化。换言之,在传统中国社会中,"礼"作为规则系统的持久性远超其物质表达的持久性。这一社会层面的观察为"规则驱动型建筑"概念提供了更广阔的文化语境支撑:宗祠建筑的空间逻辑,正是这一普遍社会规则系统在建筑领域的具体投射。

\paragraph{(3)"规则-形式转译结构"作为分析单元的理论论证}

基于上述比较分析,前期研究进一步提出:对于规则驱动型建筑,类型分析的核心单元不应是"形式"本身,而应是"规则-形式转译结构"——即社会规则如何被编码为空间形式的稳定模式。

这一命题构成对罗西"持久性"概念的重要补充。罗西关于持久性的原初论述强调的是"形式持久,功能流变",即建筑的物质形式能够超越具体功能的变迁而保持稳定,这是他"推进性持久性"的核心含义。然而,对于规则驱动型建筑,情况恰恰相反:形式本身可能并不独特甚至相当灵活,真正持久的是社会规则转译为空间形式的生成逻辑。广府宗祠的平面形制与普通民居难以区分,但"昭穆制度决定神主排列""祭祀程序编排仪式动线"这些转译规则却跨越数百年保持稳定。因此,前期研究提出补充命题:对于规则驱动型建筑,转译结构持久,形式可变。

这一命题同时为罗西"记忆装置"的隐喻提供了操作化路径。罗西将建筑比作记忆装置,但这一比喻始终停留在隐喻层面,缺乏具体的分析手段。前期研究尝试将其操作化为:建筑是"规则的编码装置"。所谓编码,是指社会规则通过特定的空间手段被物质化的过程。通过对中国宗祠、书院、庙宇等礼制建筑文献的考察,前期研究初步归纳了三种核心编码机制:其一,等级规则编码为轴线位置,尊者居中、卑者居侧;其二,禁忌规则编码为边界设置,内外有别;其三,仪式规则编码为动线编排,进退有序、行止有节。这三种编码机制在不同地区、不同时代的礼制建筑中反复出现,显示出较强的跨案例稳定性,为"规则驱动型建筑"概念的提出奠定了经验基础。

\subsubsection{多模态分析框架的方法论整合}

\paragraph{(1)既有方法的适用性评估}

前期研究对可能用于建筑空间模式分析的既有方法进行了系统评估。

符号学方法为理解建筑意义生成提供了另一条路径。皮尔斯符号学的符号-对象-解释项三元关系框架尤其适用于规则驱动型建筑分析:空间元素是符号,礼制规则是对象,使用者的理解是解释项。这一框架有效地将"形式"与"规则"置于统一的分析视野中,为"编码-解码"分析路径提供了理论支撑。前期研究发现,符号学视角能够很好地解释为何广府宗祠在视觉上与民居相似,却能被当地居民准确识别——识别的依据不是形式特征,而是对编码规则的文化理解。

基于知识图谱的多跳推理技术作为知识表征与推理的手段,为"规则-形式转译结构"的形式化表达提供了可能。CIDOC CRM是文化遗产领域的国际本体标准,其"事件中心"建模范式强调行为、时间与物质载体的关联,与规则驱动型建筑的"行为-空间"关联分析具有良好的契合度。前期研究认为,通过设计领域本体扩展,可以将礼制规则、宗族角色、仪式行为与空间单元纳入统一的知识表征框架。

\paragraph{(2)方法整合的理论框架}

基于上述评估,前期研究提出了整合符号学、空间句法与知识图谱的方法论框架:

\begin{figure}[htbp]
    \centering
    \begin{tikzpicture}[
        % 全局样式定义
        node distance = 2cm, 
        block/.style = {
            align=center, 
            font=\sffamily\small,
            inner sep=6pt,
            % 如果需要边框,取消下面一行的注释
            % draw=black, thick 
        },
        arrow/.style = {
            ->, 
            >=Stealth, 
            thick, 
            draw=black
        },
        line/.style = {
            thick, 
            draw=black
        }
    ]

        % --- 第一层 ---
        \node [block] (top) {理论层:类型学分析框架 $\rightarrow$ 规则驱动型建筑分析框架};

        % --- 定义第二层的基准高度 (分叉点) ---
        % 我们先在第一层下方 1cm 处设一个分叉点
        \coordinate (fork_top) at ([yshift=-1cm]top.south);
        
        % --- 第二层 ---
        % 节点现在定位在分叉点下方再往左右偏 4cm 的位置,anchor=north 确保它们顶部对齐
        \node [block, below left=1cm and 0.5cm of fork_top, xshift=-3.5cm, anchor=north] (left1) {形式驱动分析\\(符号学视角)};
        \node [block, below right=1cm and 0.5cm of fork_top, xshift=3.5cm, anchor=north] (right1) {规则驱动分析\\(人类学视角)};

        % --- 第三层 ---
        % 增加垂直间距到 2.2cm
        \node [block, below=2.2cm of left1] (left2) {视觉识别与量化\\(语义分割 + 量化空间分析)};
        \node [block, below=2.2cm of right1] (right2) {规则抽取与形式化\\(文献诠释 + 知识图谱 + 质性空间分析)};

        % --- 第四层 ---
        % 放在整张图的中轴线上
        \node [block, below=7.5cm of top.south] (bottom1) {多模态 AI 整合\\(跨模态语义关联与推理)};

        % --- 第五层 ---
        \node [block, below=1.2cm of bottom1] (bottom2) {知识发现与验证};

        % ================= 连线逻辑 (修复箭头逻辑) =================

        % 1. 修复:第一层到第二层的“门型”分叉箭头
        \draw [line] (top.south) -- (fork_top); % 垂直向下
        \draw [line] (left1.north |- fork_top) -- (right1.north |- fork_top); % 水平横跨
        \draw [arrow] (left1.north |- fork_top) -- (left1.north); % 垂直向下指左节点
        \draw [arrow] (right1.north |- fork_top) -- (right1.north); % 垂直向下指右节点

        % 2. 第二层到第三层的垂直箭头
        \draw [arrow] (left1) -- (left2);
        \draw [arrow] (right1) -- (right2);

        % 3. 第三层到第四层的汇聚箭头 (指向多模态AI整合)
        \draw [arrow] (left2.south) -- (bottom1.north);
        \draw [arrow] (right2.south) -- (bottom1.north);

        % 4. 第四层到第五层
        \draw [arrow] (bottom1) -- (bottom2);

    \end{tikzpicture}
    \caption{整合方法论框架}
    \label{fig:framework}
\end{figure}

这一框架的理论逻辑可以从三个层面来理解。在理论层,类型学分析框架提供了"类型作为集体记忆载体"的基本框架,而规则驱动型建筑分析框架则补充了其对形式模糊型建筑的解释力,两者共同构成研究的理论基础。在方法层,框架建立了"形式驱动-规则驱动"双轨分析路径:形式驱动分析基于符号学视角,通过视觉识别与量化(语义分割+复化空间分析),回答"建筑呈现为何种形式"的问题;规则驱动分析基于人类学视角,通过规则抽取与形式化(文献诠释+知识图谱+质性空间分析),回答"形式为何如此配置"的问题。在整合层,多模态AI整合技术作为技术中介,实现跨模态语义关联与推理,最终支撑知识发现与验证。

这一框架的核心创新在于打通了"描述"与"解释"之间的方法鸿沟。传统的建筑形态研究擅长描述建筑"是什么样",但难以解释"为什么是这样";传统的礼制研究擅长解释规则的文化意义,但难以量化其空间表达。双轨分析路径使两类研究能够相互验证、相互补充:形式驱动分析发现的空间模式,可以通过规则驱动分析获得文化与礼制层面的解释;规则驱动分析抽取的礼制逻辑,可以通过形式驱动分析获得空间形态层面的验证。











\paragraph{(3)关键技术路径的初步论证}

前期研究对框架涉及的关键技术路径进行了可行性论证,确认了技术实现的基本可能性。

在视觉-语义关联方面,研究拟通过深度学习语义分割技术从建筑图像中识别空间元素,并建立与知识图谱实体的映射关系。前期文献调研表明,基于卷积神经网络与Transformer架构的语义分割模型在建筑图像分析中已有成功应用,能够有效识别建筑构件与空间区域,技术成熟度较高。针对岭南传统建筑的特殊性,需要设计专门的分割类别体系,并进行针对性的模型微调,但这属于工程实现层面的问题,不存在原理性障碍。

在文本-知识抽取方面,研究拟利用大语言模型从礼制文献中抽取结构化知识,并转化为知识图谱的实体与关系。前期小规模实验表明,大语言模型对古汉语文献具有一定的理解能力,能够识别族谱、方志中的人物关系、仪式程序与空间描述。但由于礼制知识的专业性与文献表达的复杂性,完全自动化的抽取尚难以保证准确率,需要建立"模型初抽取+专家校验"的混合工作流程。这一方案在技术上可行,但需要投入较多的人工校验成本。

在图谱-推理整合方面,研究拟基于检索增强生成技术实现知识图谱与大语言模型的整合,支持面向规则的语义推理与知识发现。前期原型开发验证了这一技术路径的可行性:通过将知识图谱中的三元组转化为大语言模型可理解的上下文,模型能够回答诸如"按照昭穆制度,某人的神主应置于何位"之类的推理问题,并能够溯源推理路径。这一能力对于验证"规则-形式转译结构"的解释力具有重要价值。

%3.2 视觉框架 ~~~~~~~~~~~~~~~~~~~~~~~~~~~~~~~~~