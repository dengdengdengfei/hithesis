\subsection{主要研究内容}

本论文将围绕"规则驱动型建筑"的类型学理论建构与多模态人工智能分析方法开发两条主线展开,以广府宗祠为实证案例。论文共分为六章,各章研究内容如下:


% ---------------------------------------------------------------------
% 4.1 主要研究内容
% ---------------------------------------------------------------------

\subsubsection{第一章 绪论}
本章将阐明研究的问题意识与学术定位。首先,将从建筑类型学的认识论困境出发,分析罗西类型学的核心贡献与隐含假设,指出其对"形式模糊型"建筑关注不足的理论空白。其次,将以中国传统民间建筑、尤其是广府宗祠为例,论证"形式模糊型"建筑对既有类型学理论的挑战,提出核心研究问题:如果形式不具备独特性,类型的持久性何以可能?再次,将明确本研究的理论目的、方法目的与实证目的,阐述研究的理论意义、方法意义与实践意义。最后,将系统梳理建筑类型学、空间分析方法、符号学与建筑、人工智能与文化遗产、华南宗族与礼制等领域的国内外研究现状,定位本研究的切入点与创新空间。

\subsubsection{第二章 理论建构:罗西类型学的拓展与规则驱动型建筑}
本章将完成核心理论建构工作。
\begin{enumerate}[label=(\arabic*)]
    \item 第一节将系统阐释罗西类型学的核心命题与方法论遗产,包括"类型作为集体记忆的载体""持久性"(Permanence)概念以及"记忆装置"(Apparatus)概念,分析罗西理论的原创性贡献,揭示其对建筑空间模式分析的启发意义。
    
    \item 第二节将界定罗西类型学的适用边界与理论空白,论证其基于西方经验的预设——类型的可识别性依赖于形式的独特性与持久性——在面对"形式模糊型"建筑时的解释力不足。本节将指出:依赖单一视觉形态的分析路径无法有效表征此类建筑的空间模式,"记忆如何编码为形式"这一分析维度的缺失呼唤新的方法范式。
    
    \item 将提出"规则驱动型建筑"(Rule-driven Architecture)的核心概念,将其界定为一类形态生成主要受社会规则(礼制、禁忌、等级)支配的建筑类型,与"形式驱动型建筑"形成对照。本节将论证:对于规则驱动型建筑,转译结构持久而形式可变,其空间模式的完整表征必须同时处理视觉形态与文化规则两个维度,这构成了多模态解析框架的理论必要性。本节还将"记忆装置"概念操作化为"规则的编码装置",提出"等级→轴线、禁忌→边界、仪式→动线"的编码机制假说,为跨模态语义关联提供理论模型。
    
    \item 第四节将论证多模态人工智能解析框架的理论基础与技术整合逻辑。将引入皮尔斯符号学的"符号-对象-解释项"三元关系,建立"视觉形态-文化规则-语义解读"的分析框架,为视觉语义分割与知识图谱的整合提供符号学依据;将拓展空间句法理论,从"物理可达性"发展到"伦理许可性",提出"伦理句法"概念,为跨模态推理中的身份-空间关联提供理论支撑;将论证知识图谱作为"规则-形式转译结构"形式化表征工具的适用性,阐明其在实现从建筑形态到文化规则系统化解码中的关键作用。
    
    \item 将论证广府宗祠作为规则驱动型建筑典型案例的选择依据,包括其形式模糊性、规则主导性、资料丰富性与典型性,说明该案例对于验证多模态解析框架有效性的示范价值,明确案例研究的预期目标。
    
\end{enumerate}

\subsubsection{第三章 基于 CIDOC CRM 的岭南传统建筑知识库构建}
本章将完成知识表征框架设计与知识库构建工作。

\begin{enumerate}[label=(\arabic*)]
    \item 第一节将阐述知识库构建的理论基础,分析 CIDOC CRM (ISO 21127) 作为上层本体的理论定位与事件中心 (\textit{Event-centric}) 建模范式,论证其与规则驱动型建筑知识表征的契合度。
    \item 第二节将完成领域本体设计工作。
    \begin{enumerate}[label=(\alph*)]
        \item 将设计核心类体系,包括建筑实体、空间单元、建筑构件、宗族角色、仪式活动、礼制规则、转译关系等,建立与 CIDOC CRM 核心类(\texttt{E22 Man-Made Object}、\texttt{E53 Place}、\texttt{E21 Person} 等)的映射关系。
        \item 设计核心关系体系,包括空间关系 (\texttt{locate\_in}, \texttt{adjacent\_to})、规则关系 (\texttt{govern}, \texttt{restrict})、转译关系 (\texttt{encode\_as}) 等。
    \end{enumerate}
    \item 第三节将开发多源数据融合与知识抽取流程。整合结构化、半结构化与非结构化三类数据来源。开发礼制文献的 LLM 辅助实体关系抽取方法;实现视觉数据"语义分割 $\rightarrow$ 构件实体 $\rightarrow$ 图谱节点"的转化流程。
    \item 第四节将完成知识库的技术实现与质量验证。采用图数据库与 RDF 三元组存储相结合的技术方案,建立本体一致性检查与领域专家评审机制。
\end{enumerate}

\subsubsection{第四章 形式驱动的人工智能分析框架}
本章将开发视觉识别与空间模式提取的技术框架。

\begin{enumerate}[label=(\arabic*)]
    \item 第一节将阐明形式驱动分析框架在整体方法体系中的定位——对应符号学的"符号识别"环节,提取空间模式的形式表征。
    \item 第二节将完成语义分割模型的设计与训练。将分割体系分为文本类(匾额、楹联等)、图像类(木雕、石雕等)、结构类(梁架、柱础等)三大类别。
    \item 第三节将开发空间模式提取方法。分析各类元素的面积占比与空间分布规律。重点验证"文本中心"现象:量化文本类元素物理占比与语义权重的关系,从符号学角度解释文本作为"指示符号"的功能。
    \item 第四节将探索空间句法的"伦理句法"拓展。应用传统空间句法完成拓扑分析后,引入"身份维度",构建不同身份(长幼、男女、嫡庶等)的差异化可达性图谱。
\end{enumerate}

\subsubsection{第五章 规则驱动的人工智能分析框架与知识发现}
本章将开发礼制逻辑表征与知识发现的技术框架,完成理论验证工作。

\begin{enumerate}[label=(\arabic*)]
    \item 第一节将阐明规则驱动分析框架在整体方法体系中的定位——对应符号学的"意义解码"环节,验证"规则驱动型建筑"理论假说。
    \item 第二节将完成礼制知识抽取与形式化工作。建立转译结构的形式化表达体系,如:$等级 \rightarrow 轴线位置$($\langle 角色, \text{has\_rank} \rangle \rightarrow \langle 空间, \text{locate\_at} \rangle$)等。
    \item 第三节将构建基于 RAG(检索增强生成)的语义推理系统。整合知识库检索、大语言模型生成与推理路径溯源,设计事实查询、规则推理、情境判断、反事实推理四类任务。
    \item 第四节将完成知识发现与案例深度分析。验证核心发现一:转译结构的稳定性;验证核心发现二:伦理句法的有效性。完成典型案例在仪式、等级、禁忌边界等方面的深度推演。
    \item 第五节将完成框架整合与理论回应。评估本研究对罗西类型学拓展的验证效果,评估配套方法体系(符号学解释力、空间句法拓展可行性等)的整体效果。
\end{enumerate}

\subsubsection{第六章 结论与展望}
本章将总结研究成果并展望未来方向。将从理论、方法、实证三个维度总结研究结论。凝练理论创新、方法创新、数据与应用创新点。客观分析研究局限,如案例地域覆盖、礼制文献依赖性等。最后从理论深化、方法拓展、应用迁移三个方向展望未来研究。















% ---------------------------------------------------------------------
% 4.2 实施方案及其可行性论证
% ---------------------------------------------------------------------
\subsection{实施方案及其可行性论证}

\subsubsection{技术路线}
本研究将构建"视觉语义分割—知识图谱建模—跨模态推理"三位一体的多模态人工智能解析框架,实现从建筑形态到文化规则的系统化解码:

\begin{figure}[h]
\centering
\begin{tikzpicture}[
    node distance=1.5cm and 2cm,
    block/.style={rectangle, draw, fill=blue!10, text width=5cm, text centered, rounded corners, minimum height=1cm},
    smallblock/.style={rectangle, draw, fill=green!10, text width=4cm, text centered, rounded corners, minimum height=0.8cm},
    arrow/.style={->, >=stealth, thick}
]

% 顶层框架
\node [block] (framework) {多模态人工智能驱动的解析框架};

% 三个核心模块
\node [smallblock, below left=of framework, xshift=-2cm] (vision) {视觉语义分割\\(形式表征)};
\node [smallblock, below=of framework] (knowledge) {知识图谱建模\\(规则表征)};
\node [smallblock, below right=of framework, xshift=2cm] (reasoning) {跨模态推理\\(语义关联)};

% 具体技术
\node [smallblock, below=of vision] (vision-detail) {建筑图像 $\rightarrow$\\空间元素识别};
\node [smallblock, below=of knowledge] (knowledge-detail) {礼制文献 $\rightarrow$\\转译结构形式化};
\node [smallblock, below=of reasoning] (reasoning-detail) {图谱+LLM $\rightarrow$\\RAG语义问答};

% 案例验证
\node [block, below=3cm of knowledge, fill=orange!10] (validation) {广府宗祠案例验证};

% 箭头连接
\draw [arrow] (framework) -- (vision);
\draw [arrow] (framework) -- (knowledge);
\draw [arrow] (framework) -- (reasoning);

\draw [arrow] (vision) -- (vision-detail);
\draw [arrow] (knowledge) -- (knowledge-detail);
\draw [arrow] (reasoning) -- (reasoning-detail);

\draw [arrow] (vision-detail) -- (validation);
\draw [arrow] (knowledge-detail) -- (validation);
\draw [arrow] (reasoning-detail) -- (validation);

\end{tikzpicture}
\caption{多模态人工智能驱动的解析框架技术路线图}
\label{fig:technical-route}
\end{figure}

\subsubsection{研究方法}
\begin{tabularx}{\textwidth}{@{} p{2.5cm} p{3cm} L @{}}
\toprule
\textbf{研究阶段} & \textbf{核心方法} & \textbf{技术手段} \\ \midrule
理论建构 & 文献分析、概念演绎 & 类型学批判性阅读;"规则驱动型建筑"理论框架构建 \\ \addlinespace
知识库构建 & 本体建模、知识抽取 & CIDOC CRM本体扩展;基于大语言模型的自动化知识抽取 \\ \addlinespace
形式分析 & 语义分割、空间句法 & 深度学习模型训练(SAM/DeepLabV3+);拓扑计算与"伦理句法"建模 \\ \addlinespace
规则分析 & 符号学解码、图谱推理 & 知识图谱推理;基于RAG架构的大语言模型多跳推理 \\ \addlinespace
案例验证 & 田野调查、专家评审 & 实地测绘与影像采集;跨案例对比验证;专家知识交叉验证 \\
\bottomrule
\end{tabularx}

\subsubsection{可行性论证}

\textbf{理论可行性:}罗西类型学为本研究提供理论起点,皮尔斯符号学与空间句法提供方法论资源,华南宗族研究提供人类学积淀。三者的整合具有内在逻辑一致性,"规则驱动型建筑"理论作为补充性框架,能够有效衔接类型学传统与东方建筑特质,理论路径清晰可行。

\textbf{技术可行性:}语义分割、知识图谱、大语言模型三项核心技术均已成熟,在文化遗产数字化领域已有成功应用先例(如敦煌壁画分析、古建筑病害检测等)。本研究的技术整合路径清晰,各模块间接口明确,且已具备高性能计算资源(4×RTX 4090)与跨学科合作平台(北航计算机学院、莱顿大学数学系),技术实施风险可控。

\textbf{数据可行性:}广府地区宗祠建筑保存完整、案例充足,族谱方志等历史文献资料丰富且可获取性强。前期田野调查已积累初步数据基础,建立了与地方文保部门、宗族组织的良好合作关系,后续数据采集渠道畅通,样本代表性有保障。