系统梳理国内外相关研究现状是本研究的重要基础工作。通过对已有研究成果的深入分析,可以明确学科发展脉络、识别理论空白、把握技术前沿,为本研究的理论构建和方法选择提供坚实支撑。

本研究聚焦传统建筑文化遗产的解读与传承,涉及形式认知、空间解析、文化阐释、技术赋能等多个层面,因此需要整合多学科的理论与方法。建筑类型学为传统建筑的分类与形式认知提供基础框架;空间分析方法为量化解读建筑空间特征提供技术工具;符号学理论为揭示建筑背后的文化意义提供解码路径;人工智能技术则为大规模文化遗产数据的处理与模式识别提供前沿手段。这四个研究方向共同构成了从"形式—空间—意义—技术"的完整研究链条。

本章采用"国外—国内—综合分析"的逻辑框架展开。国外研究现状重点梳理上述四个方向的基础理论与方法论体系,把握国际学术前沿动态;国内研究现状则聚焦物质遗产保护实践、传统建筑文化机制解读以及智能化技术的本土化应用,凸显中国学者的独特贡献;最后通过综合简析,识别现有研究的优势与不足,明确本研究的创新切入点。


%===========================================

% 2.1 国外研究现状

%===========================================
\subsection{国外研究现状}



\subsubsection{建筑类型学理论研究}

在建筑学的理论体系中,类型学(Typology)不仅是知识传承的基石,也是理解建筑形式生成逻辑的关键。对于探究“规则驱动型建筑”及其空间模式的研究而言,梳理类型学从“静态分类”向“动态生成”的演变过程至关重要。西方建筑类型学的发展并非一成不变,而是在不同的历史时期呈现出截然不同的关注点。本节将依据时间脉络,梳理从启蒙时代至今的类型学理论演变,以揭示传统理论在处理复杂建筑形式时的局限性,以及在人工智能时代进行理论更新的必要性。

(1) 18-19世纪:类型学概念的起源与两种不同的定义路径

现代建筑类型学的理论起源于18世纪的启蒙运动时期,当时为了整理繁杂的建筑知识,形成了两种截然不同的定义路径,这两条路径奠定了后世对“类型”理解的基础。

第一条路径侧重于\textbf{哲学思辨},以昆西(Quatremère de Quincy)为代表。他在《建筑学历史词典》(1832)\cite{quatremere1832}昆西(Quatremère de Quincy) 在《建筑学的历史辞典》(Dictionnaire historique d'architecture)中提出,必须区分“类型(Type)”与“模型(Model)”。他认为,“模型”是需要被精确复制的具象物体,而“类型”则是一种模糊的、抽象的“原始原则(Original Principle)”。例如,尽管不同地区的住宅形式各异,但它们可能共享同一个内在的生成逻辑。为了解释这种逻辑,Quatremère 提出了“性格”理论,认为建筑形式受到自然法则(本质性格)、气候与风俗(偶发性格)以及具体用途(相对性格)的共同影响。这一观点强调“规则”先于“形式”存在,为本研究探讨建筑背后的深层规则提供了早期的理论支持。

第二条路径侧重于\textbf{实用分类},以杜兰(J.N.L. Durand)为代表。作为巴黎综合理工学院的教授,Durand 为了快速培养工程师,建立了一套基于网格(Grid)和轴线的设计方法\cite{durand_method}Précis des leçons d'architecture données à l'École polytechnique。他不再纠结于抽象的哲学起源,而是将建筑分解为墙、柱等基本几何元素,通过标准化的组合规则来生成建筑。杜兰制作了著名的《Précis des leçons d'architecture données à l'École polytechnique》,将类型视为一种类似于生物学的“分类目录”。这种方法虽然极大地提高了设计的效率,但也使得“类型”逐渐变成了一种形式操作的工具,忽略了建筑背后的文化含义。

(2) 20世纪上半叶:工业化背景下“标准”对“类型”的替代

进入20世纪,随着工业革命的深入,现代主义运动试图切断与传统历史的联系。在这一时期,传统的“类型”概念被工业生产逻辑下的“标准(Standard)”所取代。

勒·柯布西耶(Le Corbusier)是这一转变的关键人物。他在《走向新建筑》\cite{lecorbusier_towards}[法]勒·柯布西耶(Le Corbusier)《走向新建筑》Vers Une Architecture中提出“住宅是居住的机器”,认为建筑应当像汽车一样,经过工业化的筛选和进化,最终形成一种通用的“标准件”。他提出的 Dom-Ino 住宅系统就是一个典型的例子,它剥离了所有的地域特色,只保留了最基本的结构框架,旨在全球范围内进行大规模复制。

与此同时,德国包豪斯学派进一步将类型学推向了科学化。瓦尔特·格罗皮乌斯(Walter Gropius)认为,建立标准类型是现代社会的必然需求,设计应服务于大众福利\cite{gropius_scope}W Gropius, The new architecture and the Bauhaus。在汉内斯·迈耶(Hannes Meyer)时期,类型设计被简化为对光照、通风等生理需求的严格计算,“最小生存限度”住宅成为了研究热点。虽然这一时期的研究极大地推动了建筑的工业化进程,但由于过分强调功能和效率,导致了战后大量城市建筑千篇一律,丧失了地域特色和文化记忆。

(3) 20世纪60-70年代:对功能主义的批判与历史类型的回归

面对现代主义城市建设造成的文脉断裂,20世纪60年代兴起的新理性主义(Neo-Rationalism)运动,主张重新通过类型学来连接历史与城市。

阿尔多·罗西(Aldo Rossi)在《城市建筑》(1966)\cite{rossi_arch_city}Aldo Rossi, The Architecture of the City一书中,严厉批判了“形式追随功能”的观点。他指出,许多古老建筑的功能在几百年间不断变化,但其形式却一直存在,这证明形式比功能更具有持久性。罗西重新强调了类型作为“逻辑原则”的重要性,并提出了“集体记忆”的概念,认为城市中的重要建筑(纪念物)承载了市民的共同记忆。他的理论核心在于:类型学不是复古,而是提取那些跨越时间、能够维系城市结构的深层形式。

安东尼·维德勒(Anthony Vidler)后来将这一阶段总结为“第三类型学”\cite{vidler_third_typology}Anthony Vidler ,The architectural uncanny: Essays in the modern unhomely。他认为,第一阶段模仿自然(启蒙时代),第二阶段模仿机器(现代主义),而这一阶段则是模仿“城市”本身。建筑师开始将城市视为一个既有的生态系统,设计的目标是延续城市的纹理。

然而,罗西等人的理论主要基于欧洲古城的砖石建筑经验,强调清晰的几何实体。对于中国岭南等地区“形式模糊”但“关系稳定”的传统建筑,这种基于固定几何形态的类型学理论往往显得解释力不足。

(4) 20世纪末至21世纪初:从形态拼贴到参数化设计的转变

20世纪70年代末,随着后现代主义的兴起,类型学一度面临危机。拉斐尔·莫内欧(Rafael Moneo)在《关于类型学》(1978)\cite{moneo_typology}Rafael Moneo, Sobre la noción de tipo 中指出,类型逐渐被异化为可以随意拼贴的视觉符号,失去了其作为深层结构的力量。

进入数字时代后,计算机技术的发展彻底改变了类型学的定义。扎哈·哈迪德(Zaha Hadid)事务所的帕特里克·舒马赫(Patrik Schumacher)提出了“参数化主义”\cite{schumacher_parametricism}Patrik Schumacher, Parametricism: A new global style for architecture and urban design ,主张否定传统的方、圆等几何分类。他认为,利用计算机算法(如拓扑几何),可以实现建筑形式的连续变化。在这种视角下,所有的建筑形式都是同一个连续体变形的结果,而不是一个个独立的“类型”。这种方法虽然在形态上极具创新,但也因为过于关注形式游戏而受到批评,被认为忽略了建筑的社会文化意义。

(5) 当前趋势:人工智能与算法驱动的生成式类型学

近年来,随着人工智能(AI)和大数据技术的发展,类型学研究正在经历新的变革。这一阶段的特征不再是寻找静态的分类,而是探索动态的生成规则\cite{ai_arch_gen}Computer Vision-based Analysis of Buildings and Built Environments: A Systematic Review of Current Approaches。

首先,机器学习技术正在革新我们对建筑分类的认知。通过深度学习算法,计算机可以处理海量的建筑图像和平面图,从中发现人类肉眼难以察觉的规律\cite{ai_arch_gen}Deep Learning Architect: Classification for Architectural Design through the Eye of Artificial Intelligence。这种“数字分类学”不再依赖理论家的主观定义,而是基于大数据的统计结果。

其次,进化算法的应用使得设计从“绘图”转向了“培育”。研究者利用算法规则(基因型)来生成建筑形式(表现型)\cite{ai_arch_gen},Genetic Algorithms Application in Urban Morphology Generation模拟生物进化的过程。在这种模式下,类型不再是一个固定的模板,而是一套可以适应环境变化的动态规则。

此外,针对非正规城市(如城中村、贫民窟)的研究也发现,这些看似混乱的建筑实则遵循着“增量生长”的内在逻辑\cite{ai_arch_gen}Towards a morphogenesis of informal settlements。这种关注“过程”和“规则”的视角,与本研究关注的岭南传统建筑具有很高的相似性——它们往往没有固定的形式,但却遵循着稳定的适应性规则。




%================================

% 2.1.2~~~~~~~~~~~~~~

%================================



\subsubsection{空间分析方法研究}

建筑空间分析方法的演进,本质上是人类对空间认知的不断深化过程。从早期的几何描述到当代的算法生成,每一次分析工具的革新,都伴随着对“空间是什么”这一本体论问题的重新回答。本节将依据时间脉络,梳理西方建筑空间分析方法从几何理性主义、拓扑结构主义向大数据与人工智能范式转型的完整图景,并分析既有方法在面对非正规、形式模糊的传统建筑空间时的适用性与局限性。

(1) 19世纪至20世纪中叶:从几何构图到结构理性的确立

在现代科学与计算工具介入建筑学之前,西方对建筑空间的分析主要建立在欧几里得几何学与美学的理性基础之上。这一阶段的核心特征是\textbf{“视角的静态化”与“对象的实体化”},即主要关注物质实体的几何构成与视觉效果。

19世纪法国理论家欧仁·维奥莱-勒-杜克(Eugène-Emmanuel Viollet-le-Duc)是这一时期理性分析的奠基人。他对哥特建筑的研究超越了风格层面的描摹,开创了一种基于“结构理性”的分析图解法。在维奥莱-勒-杜克看来,空间并非任意的艺术创作,而是一个严密的逻辑系统。他通过“解剖(dis-membering)”与“重组(re-membering)”的绘图技术,将复杂的建筑整体分解为受力构件的逻辑组合,证明了每一个形式细节背后都存在其结构上的必要性 \cite{此处填入Viollet-le-Duc引用}Eugène-Emmanuel Viollet-le-Duc, Dictionnaire raisonné de l'architecture française du XIe au XVIe siècle。这种分析范式确立了一个重要原则:建筑形式是内在规则(结构法则)的外化。这为本研究探讨“规则驱动型”建筑提供了早期的认识论基础——即理解空间的关键在于解码其背后的生成规则,而非仅仅描述其表象。



与此同时,以艾蒂安-路易·布雷(Étienne-Louis Boullée)为代表的新古典主义建筑师,则探索了另一种基于“心理感知”的几何分析路径。布雷通过对球体、金字塔等纯粹几何体的尺度与光影控制,试图量化空间对人类情感(如崇高感)的影响 \cite{此处填入Boullee引用}Étienne-Louis Boullée Architecture, essay on art。虽然当时缺乏量化工具,但他提出的“会说话的建筑(Architecture Parlante)”概念,即形式应直接表达内部功能与精神内涵,预示了后世环境心理学对空间语义的关注。

进入20世纪初,奥地利艺术史学家阿洛伊斯·里格尔(Alois Riegl)进一步将分析视角从客体转向主体。他区分了“触觉”与“视觉”两种空间认知模式,认为空间分析不应仅停留在物理测量,还应包含观察者的感知方式 \cite{此处填入Riegl引用}Alois Riegl,  Art history and theory 。这一观点后来被诺伯格-舒尔茨(Christian Norberg-Schulz)发展为现象学的“场所精神”\cite{此处填入场所精神引用}Genius loci as a meta-concept
M Vecco - Journal of Cultural Heritage, 2020 - Elsevier。,强调空间的人本主义维度。

(2) 20世纪中叶至70年代:从静态图解到城市形态的拓扑关系

随着城市化进程的加速,单一建筑的几何分析已无法应对复杂的城市问题。这一时期,建筑空间分析的视野扩展到了城市肌理,核心转变在于\textbf{“图底关系”的发现与“形态演化”}的追踪。

18世纪诺利(Nolli)绘制的《罗马大地图》在这一时期被科林·罗(Colin Rowe)等学者重新发掘。诺利地图通过将公共建筑内部留白(作为公共空间的一部分),消解了建筑内外的物理界限,直观地揭示了城市作为公共生活容器的拓扑结构 \cite{此处填入Nolli或Rowe引用}Colin Rowe, Collage city。科林·罗据此建立了“图底分析(Figure-Ground Analysis)”理论,将城市空间抽象为“实(建筑)”与“虚(空间)”的二元关系,以此量化城市的肌理密度与连续性 \cite{此处填入Rowe引用}Colin Rowe, Transparency: literal and phenomenal。

在欧洲,城市形态学(Urban Morphology)的两大流派进一步深化了对空间演变规律的分析。以穆拉托里(Muratori)为首的意大利学派提出了“过程类型学(Process Typology)”,认为城市肌理是建筑类型在时间轴上不断适应社会需求而变异的结果 \cite{此处填入Muratori引用}G Cataldai, Saverio Muratori and the Italian school of planning typology 。以康曾(Conzen)为首的英国学派则关注地块(Plot)边界的稳定性,提出了“伯吉斯循环”等概念来描述地块内部的填充与更替过程 \cite{此处填入Conzen引用}MRG Conzen, Thinking about urban form: papers on urban morphology, 1932-1998。

这一阶段的方法论贡献在于,它开始将空间视为一个\textbf{“过程”而非静止的“结果”}。然而,无论是图底分析还是形态学研究,其对象多为具有清晰产权边界和实体形态的西方城市。对于中国岭南地区常见的“边界模糊”、“多义共生”的传统空间,这种基于二元对立(图/底、公/私)的分析工具往往显得力不从心,难以捕捉其灵活多变的内在逻辑。

(3) 20世纪70年代至90年代:从视觉直觉到空间句法的数学逻辑

20世纪70年代,随着结构主义哲学的影响和计算机技术的初步应用,空间分析迎来了一场科学革命。比尔·希利尔(Bill Hillier)创立的\textbf{空间句法(Space Syntax)与本尼迪克特(M.L. Benedikt)的等视区(Isovist)}理论,标志着分析方法从“几何形态”向“拓扑组构(Topological Configuration)”的彻底转向。

空间句法的核心假设是:空间的组构方式本身就包含了社会的逻辑。希利尔将连续的空间抽象为轴线(Axial Line)或线段(Segment),利用图论(Graph Theory)计算其拓扑关系,从而推导出“集成度(Integration)”、“选择度(Choice)”等数学指标 \cite{此处填入Hillier引用}Bill Hillier, Space syntax。




这种方法成功地将不可见的社会关系转化为可视化的数学模型,并证明了空间形态可以直接预测人流分布,而无需依赖复杂的土地利用数据 \cite{此处填入Hillier引用}Bill Hillier, Space syntax。

与此同时,可视性图分析(VGA)弥补了轴线分析在二维视场上的不足。通过计算视点在空间中的可视范围(等视区),VGA能够量化空间的“可视性”与“被视性”,从而揭示空间中的监控权力与社交互动潜力 \cite{此处填入VGA相关引用}S Bandyopadhyay, VGA-classifier: design and applications。

(4) 21世纪初至2010年代:从形态描述到环境性能与大数据的多维融合

进入21世纪,随着参数化设计与大数据的爆发,空间分析的维度被极大地拓宽。分析不再局限于形态本身,而是延伸到了\textbf{“物理环境性能”与“社会感知行为”}。

一方面,以Ladybug Tools为代表的环境模拟工具,实现了物理环境分析(光、热、风)与几何建模的无缝集成。设计师可以通过遗传算法(Evolutionary Algorithms)进行性能驱动的形式寻优(Form Finding),使空间形态自动适应环境规则 \cite{此处填入Ladybug或性能分析引用}YI Ibrahim, A methodology for modelling microclimate: A ladybug-tools and ENVI-met verification study。

另一方面,移动互联网技术催生了“社会感知(Social Sensing)”范式。通过手机信令、GPS轨迹与社交媒体数据,研究者可以将人类视为移动的传感器,直接观测空间中的真实活力与情感 \cite{此处填入社会感知引用}Y Liu,Social sensing: A new approach to understanding our socioeconomic environments。结合计算机视觉(CV)技术,机器可以自动识别街景图像中的绿视率、围合度,甚至预测街道给人带来的“安全感”或“压抑感” \cite{此处填入街景分析引用}。

(5) 当前趋势:人工智能驱动的认知智能与生成式仿真

当前(2023年至今),随着深度学习与大语言模型(LLM)的突破,建筑空间分析正迈向\textbf{“认知智能(Cognitive Intelligence)”}的新阶段。这一阶段的核心不仅仅是分析既有数据,而是利用AI理解空间的深层语义并模拟未来的社会行为。

\textbf{图神经网络(GNN)}的应用解决了传统算法难以处理非欧几里得空间数据的难题。建筑空间本质上是一种拓扑图结构(Graph-structured Data),GNN能够直接在图结构上进行学习,高效地预测交通流量、识别户型功能,甚至推理空间句法的计算结果 \cite{此处填入GNN相关引用}K Han,Vision gnn: An image is worth graph of nodes。

更具革命性的是\textbf{生成式智能体(Generative Agents)}的出现。斯坦福大学的“虚拟小镇”实验表明,这些智能体可以在数字空间中模拟出复杂的社会互动(如组织聚会、传播信息) \cite{此处填入Generative Agents引用}JM Epstein,Generative social science: Studies in agent-based computational modeling 。这为建筑学提供了一种全新的分析手段。






%2.1.3 ~~~~~~~~~~~~~~~~~~~~~~~~~~~~~~~~~~~~~~~~






\subsubsection{符号学与建筑研究:从语言学类比到算法生成的认识论演变}

建筑符号学作为理解建筑形式、意义与社会功能之间关系的核心理论框架,在过去的半个世纪中经历了深刻的认识论转型。这一转型并非线性的理论迭代,而是伴随着计算技术的介入,呈现出一种断裂式的范式转移。早期的建筑符号学研究主要依赖于索绪尔(Saussure)的语言学结构或皮尔士(Peirce)的逻辑三元组,试图在建筑形式(能指)与文化内容(所指)之间建立稳定的对应关系。然而,随着数字技术从单纯的辅助绘图工具(CAD)向参数化设计、大数据分析以及生成式人工智能(Generative AI)演进,国际学术界对符号学的讨论焦点已从静态的“再现危机”转向了动态的“信息交互”与“统计分布”。当前的国际研究现状表明,建筑符号学正在经历一场由算法驱动的本体论重构,这为重新审视传统建筑类型的规则与形式转译提供了全新的理论透镜。

(1) 前参数化与早期数字时代:符号指涉性的危机与虚拟本体论的萌芽

在数字技术全面介入建筑理论之前,以翁贝托·艾柯(Umberto Eco)和查尔斯·詹克斯(Charles Jencks)为代表的学者建立了基于结构主义的建筑符号学框架。艾柯在《建筑符号学》中提出,建筑对象不仅是功能的载体,更是文化信息的传播媒介,其意义产生于社会约定的“代码”系统之中 \cite{Eco_Semiotics}Umberto Eco, A theory of semiotics。然而,随着20世纪90年代虚拟现实(VR)与计算机图形学的兴起,这一经典的“能指-所指”二元结构遭遇了挑战。



在虚拟环境与早期的数字设计语境下,学者们开始探讨符号的“指涉性危机”。尼尔·里奇(Neil Leach)在其关于伪装(Camouflage)的研究中指出,数字时代的建筑符号不再单纯追求对现实的忠实再现(Representation),而是转向了一种基于环境适应的“同化”(Assimilation)策略 \cite{Leach_Camouflage}Neil Leach, Camouflage--Architecture: Testing the Architectural Application of Neil Leach's Camouflage Theory as a Model of Place-Identity。

(2) 参数化主义时期:作为社会通信系统的建筑与连续性场域的霸权

进入21世纪,随着脚本编写(Scripting)和参数化设计工具的普及,建筑符号学迎来了一次激进的工具化转向。这一时期的核心理论贡献来自于帕特里克·舒马赫(Patrik Schumacher)提出的“参数化符号学”(Parametric Semiology)。舒马赫主张利用参数化工具建立一种新型的空间秩序,使建筑能够像语言一样精确地传达社会功能信息 \cite{Schumacher_Parametricism}Patrik Schumacher,Parametricism: A new global style for architecture and urban design。



舒马赫的理论深受尼克拉斯·卢曼(Niklas Luhmann)社会系统论的影响,他提出建筑的唯一社会功能在于“框定交际互动”。在参数化符号学的框架下,建筑空间被视为一个复杂的通信系统,其中的每一个形态变化(如曲率、纹理、光线)都应作为特定的“能指”,精确地映射到特定的社会行为或事件模式之上 \cite{Agent_Based_Semiology}P Schumacher, The autopoiesis of architecture, volume ii: A new agenda for architecture。

(3) 第二数字转向与离散性回归:数据驱动的物质性与组合符号学

随着大数据技术和离散计算(Discrete Computation)的发展,马里奥·卡尔波(Mario Carpo)提出了“第二数字转向”(The Second Digital Turn)的概念 \cite{Carpo_Second_Digital_Turn}Mario Carpo, The second digital turn: design beyond intelligence。卡尔波指出,第一数字转向(参数化主义)追求形式的连续性;而第二数字转向则依赖于离散的大数据和统计概率,强调信息的颗粒度(Granularity)与分辨率 \cite{Carpo_Big_Data}Mario Carpo, The second digital turn: design beyond intelligence。



在这一阶段,学术界出现了一股强烈的反参数化思潮,即“离散建筑”(Discrete Architecture)。以吉尔·雷茨恩(Gilles Retsin)、何塞·桑切斯(Jose Sanchez)和丹尼尔·科勒(Daniel Koehler)为代表的学者,试图复兴一种基于“分体论”(Mereology)的符号学 \cite{Discrete_Architecture}Gilles Retsin,Discrete assembly and digital materials in architecture。分体论关注部分与整体(Part-to-Whole)的关系,主张建筑是由离散的、可识别的单元通过特定的句法规则组合而成的 \cite{Mereology_Architecture}Gilles Retsin,Discrete: reappraising the digital in architecture。

(4) 人工智能时代:机器解释项的崛起与统计类型学的诞生

近年来,随着深度学习(Deep Learning)和生成式人工智能(Generative AI)的爆发式发展,建筑符号学进入了一个全新的“统计学”阶段。列夫·曼诺维奇(Lev Manovich)提出的“文化分析”理论,为理解这一现象提供了核心框架 \cite{Manovich_AI_Aesthetics}Lev Manovich,Cultural analytics 。



这一技术进步在符号学层面引发了两个关键的理论突破:首先是“潜在空间”作为“符号空间”的发现。在生成式模型中,所有的建筑形式都被映射到一个连续的高维数学空间中。在这个空间里,意义(Semantics)与距离(Distance)是等价的 \cite{Latent_Space_Semantics}Lev Manovich,Cultural analytics 。其次是“机器解释项”(Machine Interpretant)的介入。机器取代了人类成为了符号的第一解释者 \cite{Machine_Vision_Semiotics}Lev Manovich,Cultural analytics 。深度神经网络的分层结构可以被视为一种“符号聚合”过程,从低级的像素特征逐层抽象为高级的语义符号 \cite{Neural_Network_Semiotics}Lev Manovich,Cultural analytics 。







%2.1.4 ~~~~~~~~~~~~~~~~~~~~~~~~~~~~~~~~~~~~~~~~~~~~~~~~~~~~~~~~







\subsubsection{人工智能与文化遗产研究:从几何记录到认知计算与伦理反思}



在文化遗产保护领域,数字技术的应用并非新鲜事。然而,近年来随着人工智能(AI)的介入,该领域正经历着一场从“物理保存”向“认知计算”的深刻范式转移。这一过程并非简单的工具升级,而是人类试图用机器逻辑去解码、存储乃至重构历史记忆的尝试。本节将依据技术演进的时间脉络,梳理从早期的几何建模到当代的语义理解与知识合成的完整图景,并探讨这一过程中产生的伦理挑战。

(1) 2000年代至2015年:从“哑数据”的几何记录到H-BIM的语义觉醒

在数字遗产研究的早期,行业的关注点主要集中在\textbf{“如何更精确地记录现状”}。随着激光雷达(LiDAR)和无人机摄影测量技术的普及,我们获取遗产物理形态数据的能力呈指数级增长。然而,这种“数据洪流”带来了一种深刻的“认识论危机”:我们拥有了海量的高精度点云模型,但计算机只能将其识别为三维空间中的坐标集合,而无法理解其作为“建筑实体”的历史属性。这些模型在很大程度上是“哑数据(Dumb Data)” \cite{此处填入相关引用}Takeshi Inomata,Lidar, Space, and Time in Archaeology: Promises and Challenges。



为了解决这一问题,历史建筑信息模型(H-BIM)应运而生,其核心任务是从“逆向建模(Scan-to-BIM)”转向“语义富集(Semantic Enrichment)”。这一阶段的关键突破在于引入了本体论(Ontology),试图让计算机“读懂”建筑。

(2) 2015年至2020年:从视觉特征提取到深度学习的语义理解

随着计算机视觉(CV)技术的突破,针对遗产数据的处理方式发生了质变。AI不再依赖人工定义的规则,而是开始通过深度学习自动识别建筑的构件与风格。



\textbf{a. 点云语义分割:机器视觉的精细化}

面对海量且杂乱的激光扫描点云,深度学习引入了端到端的特征学习机制。例如,图基方法(Graph-based)如 DGCNN,通过在点云中动态构建图结构来捕捉几何关系,在识别历史建筑复杂的装饰边缘时表现更佳 \cite{此处填入DGCNN引用}Machine and Deep Learning Implementations for Heritage Building Information Modelling: A Critical Review of Theoretical and Applied Research
Aleksander Gil et al., 2024。

\textbf{b. 风格识别:理性符号与统计联结的博弈}

AI识别风格主要存在两条路径:符号主义路径(以形状语法为代表,具有高可解释性 \cite{此处填入ShapeGrammar引用})Digital Excavation of Mediatized Urban Heritage: Automated Recognition of Buildings in Image Sources
Tino Mager et al., 2020与联结主义路径(以深度卷积神经网络为代表,具有强泛化能力 \cite{此处填入DCNN引用})Digital Excavation of Mediatized Urban Heritage: Automated Recognition of Buildings in Image Sources
Tino Mager et al., 2020。目前的趋势是转向混合智能(Hybrid Intelligence),通过深度学习提取特征后再输入形状语法引擎进行逻辑补全。

(3) 2020年至今:从数据孤岛到知识图谱与生成式AI的综合推理

进入2020年代,遗产保护的终极挑战转向了知识合成。如何将分散在文献、档案中的文本知识与H-BIM模型关联?知识图谱(Knowledge Graph)与大语言模型(LLM)成为核心工具。



通过检索增强生成(RAG)技术,研究者可以解决LLM的“幻觉”问题。每一条关于遗产的智能回答都可以追溯到精确的图数据库或档案记录,极大地增强了历史研究的严谨性 \cite{此处填入RAG引用}W Fan,A survey on rag meeting llms: Towards retrieval-augmented large language models。









%2.2~~~~~~~~~~~~~~~~~~~~~~~~~~~~~~~~~~~~~~~~~~~~~
\subsection{国内研究现状}

相较于国外研究侧重基础理论与方法论的构建,国内学界在传统建筑研究领域呈现出鲜明的本土化特征——既强调对中国建筑实体的深度解读,又注重挖掘其背后的社会文化机制,并在数字化浪潮中积极探索技术赋能的新路径。

针对本研究关注的"规则驱动型建筑"及其空间模式解析,国内研究成果可归纳为三个互为支撑的维度:物质形式研究为理解建筑提供了实体基础,通过对聚落形态、建筑类型、地域谱系的系统梳理,积累了丰富的"形式样本库";文化机制研究则揭示了部分形式背后的逻辑,从社会结构、仪式行为、集体记忆等角度描述了规则与形式间的关系;数字化与智能化研究为传统建筑的认知提供了技术工具,从早期的几何记录到当前的语义建模、人工智能应用,正在实现从"数据存储"向"知识推理"的跨越。

这三个维度分别对应了本研究的核心关切:形(物质形态的多样性)、意(文化规则的稳定性)与术(计算解析的可能性)。只有将三者贯通,才能完整回答"岭南传统建筑为何在形式模糊的前提下仍能保持类型认同"这一核心问题。





%=========================================

%2.2.1 ~~~~~~~~~~~~~~~~~~~~~~~~~~~~~~~~~~~~~~~~~~~~~~~~~~

%=========================================


\subsubsection{传统建筑的物质形式研究}

国内对于传统建筑物质形式的研究经历了从早期的抢救性测绘与描述性记录,向类型学、形态学以及跨学科视角的深度解析转变的过程。这一领域的积淀构成了理解中国传统建筑空间逻辑的基石,同时也为本研究探讨“规则-形式转译结构”提供了丰富的实体素材与理论参照。现有的研究主要集中在聚落形态的宏观图谱构建、建筑单体的类型学分析、地域性建筑的谱系梳理以及基于人类学与考古学视角的空间形式解读四个维度。



(1) 聚落形态的量化表征与宏观图谱构建

中国传统村落与民居的物质形态研究始于20世纪初的田野调查,并在随后的几十年中逐渐体系化。早期的研究侧重于对建筑实体的记录与类型划分,以营造学社为代表的先驱们在抗战背景下对西南民居的考察奠定了基础,而20世纪50至60年代,以刘敦桢为代表的中国建筑研究室发起了全国民居调查,其成果《中国住宅概说》等著作为后续的物质形式研究提供了重要的范式 \cite{Liu_General_Dwellings}中国住宅概说 刘敦桢。进入20世纪90年代,随着《中国传统民居建筑》等汇编著作的出版,学界对民居建筑实体的调查研究进入了全面整理阶段 \cite{Traditional_Dwellings_Compendium}中国传统民居建筑 王其钧著。

在这一基础上,当代学者开始引入更为理性的分析工具,试图透过复杂的表象形式揭示聚落的空间逻辑。王昀在《传统聚落结构中的空间概念》中,尝试突破单纯的定性描述,通过实地调查将聚落的各项指标进行定量化与类型化处理 \cite{Wang_Spatial_Concept}传统聚落结构中的空间概念 王昀。这种量化思维为理解“形式模糊型”建筑背后的数理逻辑提供了方法论支持。与此同时,刘森林在《中华聚落:村落市镇景观艺术》中,将研究视野扩展至村落市镇的整体景观,揭示了社会机制与人居观念如何共同作用于物质空间的构成途径 \cite{Liu_Landscape_Art}中华聚落 村落市镇景观艺术 刘森林著。

(2) 建筑区系划分与地域性形态谱系研究

随着研究的深入,学界逐渐意识到中国幅员辽阔,单纯的整体性描述难以涵盖各地建筑的差异性,因此,基于地理与文化的区系划分成为物质形式研究的重要路径。朱光亚借用人类学的“文化圈”理论,提出了中国古代建筑文化的这一概念,对建筑谱系进行了更为宏大的梳理 \cite{Zhu_Culture_Circles}朱光亚,建筑遗产评估的一次探索。陆元鼎在《中国民居建筑》中,创造性地将汉族民系的理论模型引入建筑研究,揭示了民系迁徙与建筑形式演变之间的深层关联 \cite{Lu_Dwellings_Lineage}中国民居建筑 (上中下,三卷合集)陆元鼎编。

特别值得注意的是,针对本研究关注的岭南及周边地区,学界已积累了丰硕的成果。华南理工大学肖大威团队从文化地理学角度出发,建立了翔实的民居数据库体系,并深入讨论了自然条件、历史境遇与宗族结构对建筑形式的影响 \cite{Xiao_Lingnan_Data}。戴志坚等出版的《闽台民居建筑的渊源与形态》,以及陆琦、唐孝祥等的《岭南建筑文化论丛》,对岭南地区的建筑渊源与形态特征进行了系统考证 \cite{Lu_Lingnan_Culture}岭南建筑文化论丛 陆琦,唐孝祥主编, 。





(3) 建造技术与材料层面的形式逻辑解析

物质形式的生成离不开建造技术与材料的制约。刘妍对四川凉山彝族地区“栋梁”崇拜的考察,挖掘了传统木构营造背后的文化思维制约 \cite{Liu_Liangshan_Timber} “栋梁之材” 与人类学视角下的凉山彝族建筑营造 刘妍 - 建筑学报, 2016。她进一步结合人类学方法解读了闽浙一带编木拱桥技术的传播,认为结构设计方法的秘密性与建造难度,结合宗族势力共同造就了匠人家族的行业垄断,从而影响了技术与形式的扩散路径 \cite{Liu_Timber_Arch_Bridges}“栋梁之材” 与人类学视角下的凉山彝族建筑营造 刘妍 - 建筑学报, 2016。



部分研究还采用了考古学的方法来剖析建筑形式的演变。吴敏运用聚落考古学理论对粤东古村落凤山楼的考察,还原了建筑形式在时间维度上的真实演变逻辑 \cite{Wu_Fengshan_Archeology}考古学研究古村落的成功尝试 许永杰 - 华夏考古, 2019 - kaogu.cssn.cn。这种基于实物遗存的深层分析,剥离了后世的干扰,为理解“规则驱动”下的形态演变提供了科学依据。








%2.2.2~~~~~~~~~~~~~~~~~~~~~~~~~~~~~~~~~~~~~~~~~~~~~~~~~








\subsubsection{传统建筑的文化机制研究}

相较于对物质形式的静态描述,国内学界对传统建筑文化机制的研究呈现出多维度的深化趋势。这一领域的探索不再满足于建筑实体的测绘与分类,而是试图揭示隐含在形式背后的社会结构、仪式行为、集体记忆与文化认同等深层动力。对于本研究提出的“规则驱动型建筑”而言,这些非物质的文化机制构成了制约形式生成的“元规则”(Meta-rules)。现有的研究成果主要集中在建筑人类学的社会结构解析、身体与仪式的空间建构、以及村落遗产的记忆与认同机制三个层面。

(1) 基于人类学视角的社会结构与空间图式

建筑人类学视角的引入,标志着国内传统建筑研究从“物”的层面转向了“人”与“社会”的深层互动。这一研究路径强调建筑不仅是物理空间的围合,更是社会关系的投射与文化观念的载体。常青院士较早将建筑人类学引入国内,指出建筑中物质的“器”与非物质的“意”之间存在辩证关系,传统建筑的形式往往是民间社会生活需求和集体无意识塑造的结果 \cite{Chang_Anthropology_Theory}基于文献计量学的国内建筑学与人类学交叉研究趋势讨论 家婷 周, 虹 陈, 瑞鑫 周, 睿 张, 镇邱 郎,Education and Teaching Innovation, 2024, 0 citations,。



在微观层面,学界积累了大量基于田野调查的案例。赵巧艳、张晓春等人对侗族建筑的研究表明,空间布局作为一种象征体系,深刻折射出深层文化内涵。例如,象征家庭延续的主屋通过子嗣繁衍和香火延续的行为规则来体现,而象征家庭关系的谷仓则以家族观念和男性地位的形式达成 \cite{Zhao_Dong_Social_Structure}侗族传统民居“象征符号”研究之三 侗族传统民居的指涉性象征符号阐释 赵巧艳。这种空间形式的差异性传递了不同的象征意涵,说明建筑形式并非随意的审美选择,而是严格遵循着社会伦理与权力结构的“规则驱动”。

(2) 身体体验与仪式行为的空间建构机制

如果说社会结构决定了空间的静态骨架,那么身体体验与仪式行为则赋予了空间动态的意义。这一维度的研究侧重于探讨人的行为活动、身体感知以及周期性的仪式如何“生产”空间。



在身体与体验层面,童先生主张从体验者的身体经验出发还原建造现场,阐释游赏者对园林的体验。仪式行为作为一种高度程式化的文化规则,对空间塑造具有决定性作用。郑静通过对福建土楼仪式行为的解读,反思了地方社区传统建造知识体系对现代建筑学语言的消解 \cite{Zheng_Tulou_Ritual}鄭靜,2014,<土樓與人口的流動:清代以來閩西南僑鄉的建築變革>. 《全球客家研究》。常青和劭陆关于“东西阶”与“奇偶数开间”的人类学解说,则深刻阐释了礼制文化如何通过特定形式得以展演 \cite{Chang_Shao_Sequence}东西阶与奇偶数开间 邵陆, 常青 - 营造第三辑(第三届中国建筑史学国际研讨会论文选辑), 2004。。

(3) 集体记忆与文化认同的遗产价值建构

在遗产保护与文化传承的语境下,国内学者广泛引入了记忆理论与认同理论,探讨传统建筑如何作为“记忆场所”维系村落共同体。这一研究路径不仅解释了建筑类型的持久性,也为形式背后的情感维度提供了支撑。



基于哈布瓦赫(Halbwachs)的集体记忆理论,吕龙等人将乡村文化记忆划分为“硬记忆”(如标志性建筑)和“软记忆”(如节日仪式),探讨了记忆表征形式与空间类型的映射关系 \cite{Lu_Memory_Mapping}乡村文化记忆资源的“文—旅” 协同评价模型与应用——以苏州金庭镇为例 吕龙, 黄震方, 李东晔 - 自然资源学报, 2020 - jnr.ac.cn。认同理论则揭示了建筑在构建地方感中的核心作用。文化认同被视为遗产存续的核心机制,孙莹、肖大威通过对梅州客家传统村落边界的研究,探讨了其所表征的宗族内部认同与外部区分体系 \cite{Sun_Xiao_Identity_Boundary}梅州客家传统村落空间形态及类型研究 孙莹, 肖大威, 徐琛 - 建筑学报, 2016。























%2.2.3 ~~~~~~~~~~~~~~~~~~~~~~~~~~~~~~~~~~~~~~~~~~~~~~~~~~~~~~~~~~~











\subsubsection{传统建筑的数字化与智能化研究}

中国传统建筑的数字化研究历程,本质上是一部技术工具与文化认知深度纠绰的演进史。自20世纪30年代营造学社引入西方测绘学方法以来,其核心诉求始终是对“行将消逝的遗产”进行抢救性记录。然而,在当前的“新工科”与数字人文双重语境下,学术界正处于从“数字化保护”(Digital Conservation)向“智能化计算”(Intelligent Computing)跨越的关键转折点 \cite{Ref1}Digitalized preservation and presentation of historical building – taking traditional temples and dougong as examples
Wun-Bin Yang, Y. Yen, Hung-Ming Cheng 2015, 3 citations。这一范式转移的标志性特征,是从关注建筑的“物理本体”转向关注建筑的“语义逻辑”。



(1) 基于时空维度的数字重构与高保真复原

早期的数字化研究主要致力于解决传统建筑信息的精准记录与几何复原问题。清华大学郭黛姮教授团队完成的“数字圆明园”项目,通过查阅《圆明园内工则例》、样式房图档及考古数据,构建了一个涵盖不同历史时期演变的“四维时空数据库” \cite{Guo_Digital_Yuanmingyuan}数字再现圆明园 郭黛姮 - (No Title), 2012 - cir.nii.ac.jp。这种基于多源考据的“数字重构”方法,实际上是一种逆向的“规则推演”过程。虽然该阶段属于“精细化手工数字化”,但其积累的海量对应关系为后续的AI模型训练提供了宝贵的“真值数据” \cite{Ground_Truth_Data} 郭先生以现代技术唤醒文化遗产 时潇, 吴更生 - 世界遗产, 2014。

在地域性记录方面,华南理工大学施瑛教授团队利用无人机倾斜摄影与GIS技术,建立了珠三角地区大量传统村落的高精度数字档案 \cite{Shi_Lingnan_Digital}华南民居研究的发展历程与当代的传承创新.潘莹, 冯思懿, 施瑛 - New Architecture, 2024 -。研究特别关注宗祠等具有宗族文化特征的空间,并尝试将宗族关系的社会学结构映射到空间结构上。

(2) 空间基因的量化表征与拓扑结构解析

随着计算技术的发展,研究者开始尝试利用量化方法提取隐藏在建筑形式背后的“空间基因”。天津大学张玉坤教授提出的“聚落空间基因”理论,主张利用GIS、空间句法等量化工具提取传统聚落的遗传密码 \cite{Zhang_Spatial_Gene}“中国传统村落” 评选及分布探析 曹迎春, 张玉坤 - 建筑学报, 2013。这种将复杂的空间形态转化为低维特征向量的过程,实质上完成了从“形式模糊”到“数理精确”的转译。



为了更深层次地表达结构逻辑,同济大学的研究团队开始探索图神经网络(GNN)在遗产数据中的应用。相比于传统方法,GNN能够更好地捕捉建筑构件之间的拓扑关系(如梁与柱的连接逻辑),将建筑空间的拓扑结构转化为AI可计算的图向量 \cite{GNN_Heritage}基于点云技术的乡村景观遗产空间信息记录与可视化方法研究 杨晨, 韩锋, 刘春 - Landscape Architecture Journal, 2018。这种方法精准对应了“规则驱动型建筑”中的“句法规则”,即建筑的本质在于要素间的连接方式与组合逻辑。

(3) 本体论驱动的语义建模与知识图谱构建

针对传统建筑特有的构造逻辑,如何让计算机“读懂”建筑成为核心议题。南京工业大学郭华瑜教授团队致力于解决HBIM(Heritage Building Information Modeling)无法原生描述中国古建复杂构件(如斗拱榫卯)的问题 \cite{Li_HBIM_Ontology}探索一种古建筑认知与表征的新方法——基于《 北平智化寺如来殿调查记》 的智化寺万佛阁大木构架数字信息模型建设研究 郭华瑜, 韩庆洁, 孙政 - 新建筑, 2025 。其核心贡献在于开发了专用的领域本体,为计算机定义了一套“语法规则”。




(4) 人工智能驱动的视觉认知与生成式设计

近年来,深度学习技术的爆发使得计算机视觉(CV)与生成式AI开始介入传统建筑研究。天津大学韩青等人利用卷积神经网络(CNN)对传统村落图像进行风格分类,并引入Grad-CAM技术生成热力图,显示AI关注的具体部位(如马头墙) \cite{Han_CNN_Classification}Towards Classification of Architectural Styles of Chinese Traditional Settlements Using Deep Learning: A Dataset, a New Framework, and Its Interpretability Qing Han, Chao Yin, Yunyuan Deng, Pei-Ling Liu Remote Sensing, 2022,。这种可解释性AI技术使得机器的视觉认知过程变得透明。

在生成式设计方面,清华大学团队探索了基于深度强化学习(RL)的AI Agent在空间规划中的应用。这种思路能够模拟传统工匠在风水、礼制等规则约束下的空间推演过程。

(5) 现有研究的“模态割裂”与多模态融合趋势

纵观国内研究现状,虽然在单一模态上取得了突破,但整体上仍存在明显的“模态割裂”现象 \cite{Modal_Fragmentation}Cross-modal representation learning and generation
Huafeng Liu, Jingjing Chen, Lin Liang, Bingkun Bao, Zechao Li, and 2 more Journal of Image and Graphics, 2023。目前的瓶颈在于缺乏一个整合框架将视觉模态(照片)、几何模态(点云、拓扑)与语义模态(文献、法则)打通。例如,目前的知识图谱多基于文本构建,缺乏对图像特征的自动关联能力。因此,构建一个“图-文-形”统一的多模态AI表征框架,实现跨模态的检索与生成,是突破当前瓶颈的关键所在 \cite{Multimodal_Future}A Survey of Full-Cycle Cross-Modal Retrieval: From a Representation Learning Perspective Suping Wang, Ligu Zhu, Lei Shi, Hao Mo, Songfu Tan Applied Sciences, 2023。






...
\subsection{国内外文献综述的简析}



通过对国内外关于建筑类型学、空间分析方法、符号学以及人工智能在文化遗产领域应用研究的系统梳理,可以发现,尽管既有研究在各自领域内取得了显著的理论与实践成果,但在构建从建筑形态到文化规则的系统化解析框架方面,依然存在理论层面的盲区与技术维度的断裂。具体而言,现有研究尚未形成一套能够整合视觉形态识别、空间拓扑解析与文化语义推理的多模态人工智能驱动方法体系,难以为广府宗祠为代表的岭南传统建筑这类规则驱动型空间模式的表征与解析提供可操作的技术范式。本节将从现有研究的主要贡献与存在的不足进行总结与评述。

\subsubsection{现有研究在理论演进与跨学科方法上的主要贡献}

纵观全球视野下的建筑学发展历程,学术界在对传统建筑的认知范式上已经完成了从“静态形式描述”向“动态逻辑生成”的关键跨越。在理论建构方面,西方建筑类型学经历了从启蒙时代的哲学思辨到现代主义的工业标准化,再到新理性主义对城市历史记忆的回归。昆西(Quatremère de Quincy)\cite{quatremere1832} The true, the fictive, and the real: the historical dictionary of architecture of Quatremere de Quincy Q De Quincy - 1999对“类型”与“模型”的区分,奠定了“规则先于形式”的认识论基础;而罗西(Aldo Rossi) \cite{rossi_arch_city}The architecture of the city A Rossi - 1984 则通过对形式持久性的论证,确立了类型学作为城市结构维系力量的地位。这些理论成果为理解传统建筑的深层结构提供了必要的哲学支点,使得研究者能够超越转瞬即逝的功能需求,去探寻建筑背后更为稳固的逻辑内核。

在空间分析的方法论层面,以希利尔(Bill Hillier)为代表的空间句法学派 \cite{hillier_social_logic} Space syntax B Hillier, A Leaman, P Stansall… - … and Planning B …, 1976实现了空间研究从“几何形态”向“拓扑组构”的科学转向。这一转变的重大贡献在于,它揭示了空间布局本身所蕴含的社会逻辑与权力秩序,证明了不可见的社会规则可以通过数学化的拓扑关系得以表征。与此同时,数字化技术的介入为文化遗产的保护与认知提供了高精度的工具。从初期的几何建模到当前的H-BIM语义富集 \cite{murphy_hbim}数字化遗产景观: 基于三维点云技术的上海豫园大假山空间特征研究 杨晨, 韩锋 - 中国园林, 2018,以及知识图谱在遗产数据中的应用,学术界已建立起了一套相对完善的物理本体记录体系。

国内研究则在地域性建筑的谱系梳理与社会机制解析上展现了显著优势。刘敦桢 \cite{liu_general_dwellings}刘敦桢[28]的《中国住宅概况》、陆元鼎 \cite{lu_dwellings_lineage}中国传统民居研究二十年 陆元鼎 - 古建园林技术, 2003 等前辈学者通过广泛的田野调查,确立了中国传统民居研究的区系范式,并将社会学、人类学的视野引入空间分析。常青院士 \cite{chang_anthropology}建筑人类学发凡 常青 - 建筑学报, 1992 提出的建筑人类学视角,深刻揭示了民间社会生活与集体无意识如何通过特定仪式行为塑造建筑空间。这些研究不仅为本课题提供了详实的实证素材,更在“形”与“意”之间建立了初步的关联,为进一步探讨文化规则对形式生成的驱动机制提供了坚实的理论土壤。

\subsubsection{现有研究在形式模糊性与规则转译中的主要不足}

尽管既有研究已取得了丰硕成果,但在面对岭南传统建筑这种高度依赖于“环境适应性规则”而非“固定几何特征”的建筑体系时,现有的理论框架与分析工具仍显露出明显的局限性。

首先,传统类型学理论在处理“形式模糊性”问题上存在严重的路径依赖。以罗西为代表的新理性主义类型学,其核心逻辑建立在欧洲古代砖石建筑的“清晰几何实体”之上。在这种范式下,类型被简化为可以识别的几何图形(如方庭、轴线、柱廊)。然而,岭南传统建筑的特质在于其形式的“离散性”与“适应性”——在宗族礼制、风水观念与高密度气候环境的共同驱动下,建筑往往呈现出一种“规则清晰但形式模糊”的特征。既有理论难以解释,为什么在不同的地形条件与地块边界下,看似迥异的形式却能共享同一种“身份认同”。这种对固定几何形态的执着,导致了在面对非标准化的传统空间时,类型学研究往往退化为简单的视觉分类,丧失了对深层逻辑的解释力。

在技术应用层面,人工智能在建筑遗产领域的研究多停留在“视觉表象”的识别与迁移上,缺乏对“生成逻辑”的深度解析。目前的深度学习模型在识别建筑风格、自动化分类以及风格迁移(Style Transfer)方面表现出色,但这种基于图像相关性的算法往往忽略了建筑生成的因果链条。现有的数字化手段往往将传统建筑拆解为孤立的构件或语义标签,难以表征“文化规则如何驱动空间拓扑演变”这一复杂的动态过程。此外,研究中普遍存在“模态割裂”现象:文本记载的营造规则、几何层面的点云数据、以及社会学维度的行为逻辑,往往存储在互不相通的数据库中,缺乏一个能够实现跨模态认知与推理的统合框架。

