
如何运用多模态人工智能技术实现传统建筑遗产的科学表征与知识传承,是当前建筑遗产保护领域面临的重要课题。本研究以岭南传统建筑尤其是广府宗祠为例,探索构建多模态人工智能驱动的建筑空间模式表征与解析框架。






%1.1 课题来源和研究背景
\subsection{研究背景}

\subsubsection{国家战略与时代需求}

党的十八大以来,以习近平同志为核心的党中央高度重视中华优秀传统文化的传承与发展。习近平总书记多次强调,要"让收藏在博物馆里的文物、陈列在广阔大地上的遗产、书写在古籍里的文字都活起来"。2017年,中共中央办公厅、国务院办公厅印发《关于实施中华优秀传统文化传承发展工程的意见》,明确提出要"加强文物保护利用和文化遗产保护传承"。2021年,中共中央办公厅、国务院办公厅印发《关于在城乡建设中加强历史文化保护传承的意见》,强调要"建立分类科学、保护有力、管理有效的城乡历史文化保护传承体系"。

与此同时,数字中国建设进入全面推进阶段。《"十四五"数字经济发展规划》将"数字技术与实体经济深度融合"作为主线,文化遗产的数字化保护与传承成为这一战略的重要组成部分。2022年,中共中央办公厅、国务院办公厅印发《关于推进实施国家文化数字化战略的意见》,明确提出要"推动文化遗产数字化保护","运用数字化手段提升文化遗产保护、传承和利用水平"。在此背景下,传统建筑遗产的数字化表征与知识传承成为学术界与实践领域共同关注的重要议题。

中国传统建筑遗产类型丰富、地域特色鲜明,数字化保护工作需要深入理解不同类型、不同地域建筑的文化特质与形制规律。在众多地域建筑体系中,岭南传统建筑因其独特的地理环境、文化传统和营造技艺,形成了有别于中原及其他地区的鲜明特征,具有重要的研究价值与保护意义。

岭南地区作为中国传统建筑遗产的重要分布区域,保存了大量具有鲜明地域特色的传统建筑。广府宗祠作为岭南传统建筑的典型代表,不仅承载着丰富的历史信息与文化记忆,更是中华优秀传统文化在建筑领域的重要载体。如何运用现代信息技术实现对这类遗产的科学表征与有效传承,既是响应国家战略的现实需求,也是建筑遗产研究领域面临的重要课题。

然而,这一实践需求的背后,隐含着一个尚未充分回应的理论问题:现有建筑类型学理论如何更好地适用于中国传统民间建筑?这一问题的回答,将直接影响遗产数字化工作的理论基础与方法路径。

\subsubsection{数字时代建筑分析方法的转型机遇}

传统建筑遗产研究长期以来主要依赖静态的形态分析方法。从梁思成先生开创的中国古建筑测绘传统,到西方建筑类型学的形态归纳方法,研究者通过实地测绘、图纸分析、类型比较等手段,记录建筑的尺度、比例、结构与装饰,建立起基于视觉形态特征的认知体系。这种方法在建筑遗产的保护与研究中发挥了重要作用,为我们留下了大量珍贵的测绘资料与类型学成果。然而,这种以视觉形态为中心的静态分析方法也存在固有的局限性。

其一,它主要关注建筑的物质形态与外观特征,对于形态背后的文化规则、社会逻辑与使用模式的揭示能力有限;其二,它高度依赖研究者的个人经验与专业判断,难以处理大规模的遗产数据,也难以实现隐性知识的显性化表达;其三,在面对形态特征不显著的建筑类型时,传统方法往往难以准确把握其类型特征与文化价值。

数字中国建设与人工智能技术的发展,为建筑遗产研究带来了方法论转型的历史性机遇。计算机视觉技术使得建筑形态的自动识别与大规模分析成为可能;知识图谱技术为建筑知识的结构化表达与关联推理提供了新的手段;空间句法的计算化发展使得空间结构的定量分析日益精细;大语言模型与多模态人工智能的兴起,则为实现视觉、文本、符号等多源信息的跨模态整合开辟了新的路径。

更重要的是,这些技术进步推动了建筑遗产研究范式的深层转变:从单一的视觉形态分析走向多模态的信息整合,从静态的形态记录走向动态的知识发现,从经验性的定性归纳走向数据驱动的定量分析与语义推理。建筑不再仅仅被看作"待测绘的对象",而是成为"待解码的知识载体"——其形态、空间、装饰、使用方式等多维度信息,都可能编码着重要的文化规则与社会逻辑。

以多模态人工智能为代表的新兴技术,为这种"解码"工作提供了强大的工具支撑。通过语义分割技术,可以将建筑图像分解为具有明确语义的构件与空间单元;通过知识图谱,可以将分散在文献、族谱、方志中的规则性知识进行形式化表达;通过检索增强生成(RAG)等技术,可以实现从视觉识别到语义理解、从形式观察到规则推理的跨模态分析。这种方法论转型,使得研究者能够从建筑的物质形态中提取更深层的文化信息,实现从"看到什么"到"理解为什么"的认知跃迁。

对于中国传统建筑遗产而言,这种方法论转型具有特殊的意义。中国传统建筑的类型特征往往不在于形式的独特性,而在于形式背后严格的营造规则、礼制规范与空间使用逻辑。如何运用数字技术将这些隐含在形态之中、却难以从形态直接读取的规则性知识显性化、可计算化,成为当前建筑遗产数字化保护的核心挑战。这既需要技术工具的创新,也需要分析范式的转变——从关注"建筑长什么样"转向追问"建筑为什么这样"。

然而,数字技术本身并不自动带来研究范式的转变。如何将这些技术工具与建筑遗产研究的理论问题、分析需求有机结合,构建一套既具有理论深度又具有技术可操作性的分析框架,仍然是一个有待回答的学术问题。这一问题的回答,需要我们深入理解研究对象的特殊性,从对象的内在逻辑出发设计方法路径。

\subsubsection{中国传统民间建筑的特殊性}

中国传统民间建筑呈现出与西方纪念碑建筑不同的形态特征。以广府宗祠为例,其基本形制多为"三间两廊"布局,与普通民居在外观上高度相似,缺乏西方纪念碑建筑那种鲜明的视觉辨识度。这种现象在中国传统民间建筑中具有普遍性——宗祠与民居、书院与住宅、会馆与商铺,往往共享相似的形式语言与空间结构。

这一特征可以概括为"形式趋同性":不同功能、不同等级的建筑在形式上呈现高度相似,类型边界不如西方建筑那样清晰可辨。这种现象并非建筑发展的"不成熟",而是中国传统建筑体系的内在特点——在统一的营造法式传统下,建筑的差异性更多地通过规模等级、装饰细节与空间使用规则来体现,而非通过截然不同的形式语言。

以广府宗祠为例,尽管其外观形式与民居趋同,但其空间配置却受到严格的宗法礼制规范。神主的安放位置、祭祀的行进路线、不同身份者的站位与活动范围,都遵循《朱子家礼》等礼制文献的规定。换言之,这类建筑的"类型性"并非主要体现在视觉形式的独特性上,而是体现在"规则-形式"转译结构的稳定性上。

这一认识为我们理解中国传统建筑的类型特征提供了新的视角:类型的持久性不一定在于形式本身,也可以在于形式背后的生成规则与转译逻辑。这种特点可以称之为"规则驱动型建筑"——其形态生成主要受社会规则(礼制、禁忌、等级)的支配,而非纯粹形式演化逻辑的主导。



%=============================

% 1.1.4 核心问题与研究命题==================================

%=============================




\subsubsection{核心问题与研究命题}

基于上述三个层面的分析,本研究面临的核心挑战可以概括为:如何在数字时代的方法论转型背景下,针对规则驱动型建筑的特殊性,构建一套能够有效实现"形式-规则-意义"跨模态解析的分析框架?这一挑战在理论、方法与技术三个层面展开为以下核心问题:

\textbf{(1)理论层面:如何理解规则驱动型建筑的类型性?}
传统建筑类型学主要基于形式的独特性与持久性来界定类型。从迪朗(Jean-Nicolas-Louis Durand)的形式分类传统,到阿尔多·罗西(Aldo Rossi)的"集体记忆"理论,类型的可识别性在很大程度上依赖于鲜明的视觉特征——如哥特教堂的尖拱、古典神庙的柱式。然而,对于形式趋同、规则驱动的中国传统民间建筑,类型的持久性体现在何处?罗西提出的"记忆装置"(apparatus)概念能否适用于这类建筑?如果形式本身不足以承载类型的可识别性,"集体记忆"是如何被编码与传承的?
这些问题要求我们在肯定既有类型学理论贡献的基础上,结合中国传统建筑的具体经验,拓展理论框架的文化适用性。

\textbf{(2)方法层面:如何构建多模态分析框架?}
规则驱动型建筑的特性决定了其类型性无法仅通过单一的视觉形态分析来把握。广府宗祠的空间配置编码着严格的礼制规范,神主的安放位置、祭祀的行进路线、不同身份者的站位与活动范围,都遵循《朱子家礼》等礼制文献的规定。理解这类建筑需要整合多种信息模态:

\textbf{视觉模态}:建筑的形态、空间、装饰等视觉信息

\textbf{文本模态}:族谱、方志、礼制文献中的规则性知识

\textbf{空间模态}:拓扑关系、轴线秩序、可达性结构

\textbf{语义模态}:构件名称、空间功能、礼制含义

如何将这些异质的信息模态有机整合,建立从视觉识别到语义理解、从形式观察到规则推理的分析路径?这需要超越传统的单一学科方法,构建跨学科的综合分析框架。

\textbf{(3)技术层面:如何实现从视觉到规则的跨模态解码?}
在技术实现层面,多模态分析框架需要解决以下关键问题:

如何将建筑图像转化为具有明确语义的结构化数据?(视觉→语义)
如何将分散在文献中的隐性规则进行形式化表达?(文本→知识)
如何建立视觉形态与礼制规则之间的关联映射?(形式→规则)
如何实现基于知识图谱的推理与问答?(知识→理解)

多模态人工智能技术的发展为这些问题的解决提供了新的可能。计算机视觉中的语义分割技术可以实现建筑图像的结构化解析,知识图谱可以对规则性知识进行本体建模,检索增强生成(RAG)技术可以整合视觉与文本信息进行跨模态推理。但如何将这些技术工具有机整合,形成适用于规则驱动型建筑的专门化分析流程,仍然是一个需要探索的技术问题。

\textbf{针对上述问题,本研究提出核心命题:}
规则驱动型建筑的空间模式表征需要超越单一视觉形态分析,构建多模态人工智能驱动的解析框架。该框架通过整合视觉语义分割、知识图谱建模与跨模态推理技术,实现从建筑形态到文化规则的系统化解码,为传统建筑的分析与表征提供可操作的方法范式。


本研究以广府宗祠为典型案例,对上述框架进行实证验证。通过构建包含实测数据、历史文献、礼制分析的综合知识库,本研究将展示多模态AI技术如何赋能规则驱动型建筑空间模式的系统化表征与深度解析,为中国传统建筑遗产的数字化保护提供理论支撑与方法示范




\subsection{研究的目的及意义}

\subsubsection{研究目的}

本研究旨在回应国家文化遗产保护与数字中国建设的战略需求,立足数字时代建筑分析方法论转型的历史机遇,针对规则驱动型建筑的特殊性,构建多模态人工智能驱动的建筑空间模式表征与解析框架。研究在理论阐释、方法构建与实证验证三个层面展开:

\textbf{(1)理论目的:阐释规则驱动型建筑的空间模式生成机制}

在理论层面,本研究旨在拓展建筑空间模式研究的认识论框架,为规则驱动型建筑的类型学研究提供新的理论视角。传统建筑类型学主要关注形式的独特性与持久性,这一理论框架对于理解西方纪念碑建筑具有很强的解释力。然而,中国传统民间建筑呈现出"形式趋同、规则驱动"的特征,其类型性主要体现在"规则-形式转译结构"的稳定性上,而非视觉形态的独特性。本研究将系统阐释"规则驱动型建筑"的概念内涵与类型学特征,论证空间模式作为文化规则物质化编码的理论命题。这一理论探索不仅有助于丰富建筑类型学的文化视野,也为理解建筑空间模式与社会文化规则之间的深层关系提供新的分析维度。同时,本研究将探讨多模态表征方法对于揭示这种深层关系的理论价值,推动建筑遗产研究从形态中心向多维度知识发现的范式转变。

\textbf{(2)方法目的:构建多模态AI驱动的空间模式表征与解析框架}

在方法层面,本研究的核心目的是构建一套系统化的多模态人工智能驱动分析框架,为规则驱动型建筑的空间模式表征与解析提供可操作的技术路径。该框架整合计算机视觉、知识工程、空间分析与自然语言处理等多学科方法,实现从视觉形态到文化规则的系统化解码。具体而言,本研究将发展基于深度学习的建筑图像语义分割方法,实现空间模式的自动化识别与结构化表征;构建基于本体标准(CIDOC CRM)的岭南传统建筑知识图谱,对礼制规则、营造法式等隐性知识进行形式化表达;整合检索增强生成(RAG)等多模态AI技术,建立视觉-文本-空间的跨模态检索与推理机制。通过这些方法创新,本研究力图建立从"看到什么"到"理解为什么"的完整分析链条,实现建筑遗产研究中定性理解与定量分析、视觉识别与语义推理、形态观察与规则解码的有机融合。这一方法体系不仅服务于本研究的具体问题,也力求形成可迁移至其他规则驱动型建筑的通用研究范式。

\textbf{(3)实证目的:以广府宗祠验证框架的有效性并构建知识资源}

在实证层面,本研究以岭南传统建筑的代表广府宗祠为典型案例,对上述理论框架与分析方法进行系统验证。广府宗祠是岭南传统建筑的代表类型,其空间配置严格遵循宗法礼制规范,是研究规则驱动型建筑的理想对象。本研究将通过实地调研、文献整理与数字化建模,构建包含形态数据、礼制文献、空间分析的综合知识库;运用多模态分析框架,系统解析广府宗祠"礼制-空间"转译的核心模式,揭示其空间模式背后的文化逻辑;实现分散在族谱、方志、礼制文献中的隐性民俗知识向显性可解析知识的转化,为岭南传统建筑遗产的价值认知、保护规划与活化利用提供数据基础与知识支撑。通过这一实证研究,本研究不仅验证方法框架的有效性,也为区域性传统建筑知识库的构建提供可参照的实践路径。

理论、方法与实证三个层面的研究目的相互支撑、有机统一。理论创新为方法构建提供认识论基础,方法构建为理论验证提供技术手段,实证研究为理论深化与方法完善提供经验支撑。通过这三个层面的系统推进,本研究力图在传统建筑表征与再现范式这一领域,实现理论突破、方法创新与应用示范的协同发展。

\subsubsection{研究意义}

\textbf{(1)理论意义}

本研究的核心理论贡献在于拓展了建筑空间模式研究的认识论框架,提出了从"形式-意义"到"形式-规则-意义"的理论路径。传统建筑类型学主要基于形式的独特性与持久性来界定类型,这一理论框架对于理解西方纪念碑建筑具有很强的解释力。然而,中国传统民间建筑呈现出"形式趋同、规则驱动"的特征——不同功能、不同等级的建筑在外观上高度相似,其类型性主要体现在空间配置背后的文化规则上。本研究提出"规则驱动型建筑"的理论概念,论证空间模式作为"规则-形式转译结构"的理论命题,揭示了类型持久性的另一种载体:不在于视觉形式本身,而在于形式背后生成规则与转译逻辑的稳定性。这一理论视角丰富了建筑学关于形式与意义关系的理论认识,为理解不同文化传统中建筑类型的形成机制提供了新的分析维度。

多模态表征方法的引入具有重要的认识论意义。本研究提出,对于规则驱动型建筑,空间模式的理解需要超越单一的视觉分析,整合视觉、文本、空间、语义等多模态信息。这种多模态表征不是信息的简单叠加,而是建立跨模态的结构化关联——视觉形态被关联到语义标签,语义标签被关联到文化规则,文化规则被关联到社会逻辑。通过这种多层次的关联网络,建筑空间模式从"静态的物质形态"转变为"动态的知识网络",从"待观察的对象"转变为"待解码的文本"。这一认识论转变为建筑遗产研究提供了新的理论基础,推动研究范式从形态记录向知识发现、从经验归纳向计算推理的转型。

本研究的理论建构过程本身也是一次中西建筑理论对话的尝试。通过将西方建筑类型学理论与中国传统建筑经验相结合,本研究探索了"以西方理论为参照、以中国经验为基础、形成本土化理论创新"的学术路径。这种对话不是简单地套用西方理论,也不是孤立地描述中国现象,而是在跨文化比较中发现理论的适用边界与发展空间,在具体经验的深入分析中提炼普遍性的理论洞见。这一研究范式对于中国建筑学的理论建设具有方法论意义,有助于推动中国建筑学界参与国际学术对话、贡献中国智慧。

\textbf{(2)方法意义}

本研究构建的多模态人工智能驱动分析框架,为建筑遗产研究提供了从静态形态分析向动态知识推理转型的技术路径。该框架整合计算机视觉、知识工程、空间分析与自然语言处理等多学科方法,实现了四个层面的方法创新:视觉表征层通过深度学习的语义分割技术,将建筑图像转化为结构化的语义数据;知识建模层通过本体标准(CIDOC CRM)的知识图谱构建,对礼制规则、营造法式等隐性知识进行形式化表达;跨模态推理层通过检索增强生成(RAG)技术,建立视觉-文本-空间的协同检索与推理机制。这一系统化框架打破了传统遗产研究中"形态分析"与"文化阐释"相互割裂的局面,实现了定量分析与定性理解、视觉识别与语义推理、形态观察与规则解码的有机融合。

隐性知识的显性化与可计算化是本研究方法创新的重要成果。传统建筑遗产中蕴含着大量隐性的民俗知识与营造智慧,这些知识往往存在于匠人口传心授的实践传统、分散在族谱方志的零散记载、隐含在建筑形态的细节之中,难以被系统记录与传承。本研究通过知识图谱与本体建模技术,将这些隐性知识进行结构化表达,使其成为可检索、可推理、可传承的显性知识资源。更重要的是,多模态AI技术使得这一过程可以部分自动化:通过语义分割自动识别建筑构件,通过自然语言处理提取文献中的规则信息,通过跨模态推理建立形态与规则的关联。这种"人工智能赋能的知识发现"模式,不仅提升了研究效率,更开辟了传统建筑知识数字化保存与活化利用的新路径,为非物质文化遗产的保护传承提供了可操作的技术方案。

\textbf{(3)实践意义}

本研究构建的多模态分析框架与岭南传统建筑知识库,为文化遗产数字化保护提供了从理论到技术的系统性支撑。在国家文化数字化战略与数字中国建设的背景下,传统建筑遗产的数字化保护已从简单的测绘记录走向深层的知识提取与智能化应用。本研究基于CIDOC CRM等国际本体标准构建的知识图谱,实现了与国际遗产信息体系的语义兼容,为跨机构、跨区域的遗产数据共享与协同研究奠定了基础。研究形成的技术流程——从图像语义分割、知识图谱构建到跨模态推理——提供了一套可复制、可推广的数字化工作范式,可为其他地区、其他类型的传统建筑遗产保护提供方法参考。更重要的是,这一框架实现了从"数字化采集"到"知识化服务"的跃迁:不仅记录建筑的物质形态,更揭示形态背后的文化逻辑;不仅建立静态的数据库,更构建支持智能检索与推理的知识系统。

本研究对于传统建筑文化的当代传承具有重要的社会价值。传统建筑遗产的保护不仅在于物质形态的保存,更在于其所承载文化内涵的阐释与传播。然而,对于普通公众而言,传统建筑的文化价值往往是隐晦的——他们能看到建筑的形态,却难以理解形态背后的礼制逻辑、等级秩序与文化意蕴。本研究通过多模态分析方法,建立了从"看得见的建筑形式"到"看不见的文化规则"的解码路径,使隐性的文化知识得以显性化、系统化。基于知识图谱的智能问答系统,可以为公众提供互动式的文化体验:通过提问了解建筑构件的功能与象征意义,通过虚拟导览体验传统祭祀仪式的空间路径,通过可视化展示理解宗法礼制的空间转译逻辑。这种"技术赋能的文化阐释"模式,有助于公众更加深入地理解传统建筑的文化价值,促进中华优秀传统文化的创造性转化与创新性发展。