\subsection{课题来源或研究背景}

\subsubsection{国家战略与时代需求}

党的十八大以来,以习近平同志为核心的党中央高度重视中华优秀传统文化的传承与发展。习近平总书记多次强调,要"让收藏在博物馆里的文物、陈列在广阔大地上的遗产、书写在古籍里的文字都活起来"。2017年,中共中央办公厅、国务院办公厅印发《关于实施中华优秀传统文化传承发展工程的意见》,明确提出要"加强文物保护利用和文化遗产保护传承"。2021年,中共中央办公厅、国务院办公厅印发《关于在城乡建设中加强历史文化保护传承的意见》,强调要"建立分类科学、保护有力、管理有效的城乡历史文化保护传承体系"。

与此同时,数字中国建设进入全面推进阶段。《"十四五"数字经济发展规划》将"数字技术与实体经济深度融合"作为主线,文化遗产的数字化保护与传承成为这一战略的重要组成部分。2022年,中共中央办公厅、国务院办公厅印发《关于推进实施国家文化数字化战略的意见》,明确提出要"推动文化遗产数字化保护","运用数字化手段提升文化遗产保护、传承和利用水平"。在此背景下,传统建筑遗产的数字化表征与知识传承成为学术界与实践领域共同关注的重要议题。

岭南地区作为中国传统建筑遗产的重要分布区域,保存了大量具有鲜明地域特色的传统建筑。广府宗祠作为岭南传统建筑的典型代表,不仅承载着丰富的历史信息与文化记忆,更是中华优秀传统文化在建筑领域的重要载体。如何运用现代信息技术实现对这类遗产的科学表征与有效传承,既是响应国家战略的现实需求,也是建筑遗产研究领域面临的重要课题。

然而,这一实践需求的背后,隐含着一个尚未充分回应的理论问题:现有建筑类型学理论如何更好地适用于中国传统民间建筑?这一问题的回答,将直接影响遗产数字化工作的理论基础与方法路径。

\subsubsection{建筑类型学的理论贡献与发展空间}

建筑类型学是20世纪建筑理论的重要成就之一。从迪朗(Jean-Nicolas-Louis Durand)的形式分类传统,到意大利建筑师阿尔多·罗西(Aldo Rossi)的"集体记忆"理论,类型学为理解建筑与城市、建筑与文化之间的关系提供了深刻的理论框架。

罗西在其经典著作《城市建筑学》(\textit{The Architecture of the City})中提出了一系列富有洞见的核心命题。他认为,类型不是抽象的形式范畴,而是"集体记忆的载体";建筑是一种"记忆装置"(apparatus),通过物质形式承载并传递社会的集体经验。罗西进一步发展了"持久性"(permanence)概念,区分了"推进性持久性"(propelling permanence)与"病理性持久性"(pathological permanence),深刻揭示了建筑形式与社会功能之间的辩证关系。这些理论洞见为建筑遗产的价值认知与保护实践提供了重要的理论支撑。

罗西类型学的形成主要基于意大利城市与西方建筑传统的经验。在其理论框架中,类型的可识别性在很大程度上依赖于形式的持久性与独特性——如哥特教堂的尖拱、古典神庙的柱式等,这些鲜明的形式特征使得类型易于识别与传承。这一理论视角对于理解西方纪念碑建筑具有很强的解释力。

然而,当我们将视野拓展至更广阔的建筑遗产领域,特别是中国传统民间建筑时,可以发现罗西类型学存在进一步发展的空间。如何将这一理论框架与非西方建筑传统相结合,使其更好地服务于多元文化背景下的遗产保护实践,成为值得深入探讨的学术问题。

\subsubsection{中国传统民间建筑的特殊性}

中国传统民间建筑呈现出与西方纪念碑建筑不同的形态特征。以广府宗祠为例,其基本形制多为"三间两廊"布局,与普通民居在外观上高度相似,缺乏西方纪念碑建筑那种鲜明的视觉辨识度。这种现象在中国传统民间建筑中具有普遍性——宗祠与民居、书院与住宅、会馆与商铺,往往共享相似的形式语言与空间结构。

这一特征可以概括为"形式趋同性":不同功能、不同等级的建筑在形式上呈现高度相似,类型边界不如西方建筑那样清晰可辨。这种现象并非建筑发展的"不成熟",而是中国传统建筑体系的内在特点——在统一的营造法式传统下,建筑的差异性更多地通过规模等级、装饰细节与空间使用规则来体现,而非通过截然不同的形式语言。

以广府宗祠为例,尽管其外观形式与民居趋同,但其空间配置却受到严格的宗法礼制规范。神主的安放位置、祭祀的行进路线、不同身份者的站位与活动范围,都遵循《朱子家礼》等礼制文献的规定。换言之,这类建筑的"类型性"并非主要体现在视觉形式的独特性上,而是体现在"规则-形式"转译结构的稳定性上。

这一认识为我们理解中国传统建筑的类型特征提供了新的视角:类型的持久性不一定在于形式本身,也可以在于形式背后的生成规则与转译逻辑。这种特点可以称之为"规则驱动型建筑"——其形态生成主要受社会规则(礼制、禁忌、等级)的支配,而非纯粹形式演化逻辑的主导。

\subsubsection{核心问题与研究命题}

基于上述分析,本研究提出以下核心问题:

\begin{enumerate}
    \item 对于形式不具备显著独特性的建筑类型,如何理解其类型的持久性与可识别性?
    \item 罗西提出的"记忆装置"概念能否适用于中国传统民间建筑?如果适用,需要何种理论补充与方法创新?
    \item 如何构建一套能够有效表征与解析"规则驱动型建筑"的分析方法体系?
\end{enumerate}

针对这些问题,本研究提出核心命题:

\begin{quote}
对于"规则驱动型建筑",类型的持久性不在于形式本身,而在于"规则-形式转译结构"的持久性。广府宗祠的类型性被编码于视觉形态之中,却无法仅从视觉形态中解码——理解它需要还原其背后的礼制生成逻辑。
\end{quote}

这一命题并非对罗西类型学的否定,而是在充分肯定其理论贡献的基础上,结合中国传统建筑的具体经验,对其进行补充与拓展。罗西敏锐地洞察到建筑作为"记忆装置"的本质,本研究则进一步追问:当形式本身不足以承载类型的可识别性时,"记忆"是如何被编码与传承的?

本研究认为,对于规则驱动型建筑,建筑是"规则的编码装置":等级被编码为轴线位置,禁忌被编码为空间边界,仪式被编码为行进动线。理解这类建筑的类型性,需要建立从形式回溯规则的分析路径。这一理论视角的提出,既回应了中国传统建筑遗产保护的实践需求,也为建筑类型学理论的发展提供了新的学术增长点。

在方法层面,这一理论视角的落实需要整合多种分析工具。符号学提供了"编码-解码"的分析视角,空间句法提供了空间结构的量化方法,知识图谱提供了规则形式化表达的技术手段,多模态人工智能则为视觉与符号的跨模态关联提供了新的可能。本研究将在这些方法资源的基础上,构建一套适用于规则驱动型建筑的多模态分析框架,并以广府宗祠为案例进行实证验证。




\subsection{研究的目的及意义}

\subsubsection{研究目的}

本研究旨在回应国家文化遗产保护与数字中国建设的战略需求,立足中国传统建筑遗产的具体经验,在理论、方法与实证三个层面实现研究目标。

\textbf{(1)理论目的}

在充分肯定罗西类型学理论贡献的基础上,结合中国传统民间建筑的特点,提出适用于"规则驱动型建筑"的类型理论,丰富和拓展建筑类型学的认识论框架。具体而言,本研究将系统阐释"规则驱动型建筑"的概念内涵与类型学特征,论证"规则-形式转译结构"作为类型持久性载体的理论命题,并将罗西的"记忆装置"概念进一步操作化,使其更好地适用于中国传统建筑遗产的分析与阐释。

\textbf{(2)方法目的}

构建一套整合符号学、空间句法与人工智能技术的多模态分析方法体系,为传统建筑遗产的数字化表征与知识发现提供可操作的工具支撑。本研究将发展空间句法的"身份维度",提出"伦理句法"的分析视角;整合知识图谱与大语言模型技术,构建"形式驱动-规则驱动"双轨分析框架;实现从视觉识别到语义推理的跨模态分析路径,形成可迁移至其他规则驱动型建筑的研究范式。

\textbf{(3)实证目的}

以广府宗祠为典型案例,验证理论框架与分析方法的有效性。本研究将构建符合国际标准(CIDOC CRM)的岭南传统建筑知识库,归纳广府宗祠"礼制-空间"转译的核心模式,实现隐性民俗知识向显性可计算知识的转化,为岭南传统建筑遗产的保护传承提供数据基础与知识支撑。

\subsubsection{理论意义}

\textbf{(1)拓展建筑类型学的文化视野}

罗西类型学的形成主要基于西方城市与纪念碑建筑的经验,其理论洞见对于理解建筑与集体记忆的关系具有普遍价值。本研究在此基础上,引入中国传统民间建筑的经验素材,探讨类型学理论在非西方建筑传统中的适用性与发展可能。这一工作不仅有助于拓展建筑类型学的文化视野,也是中国建筑学界参与国际学术对话、贡献中国经验与中国智慧的有益尝试。

\textbf{(2)提出"规则驱动型建筑"的理论概念}

本研究提出"规则驱动型建筑"这一概念,用以描述形态生成主要受社会规则支配的建筑类型。这一概念的提出,有助于建立更为细致的建筑类型分析框架,区分"形式驱动型"与"规则驱动型"两种不同的类型生成逻辑。这种区分并非对立与割裂,而是对建筑类型多样性的更加充分的认识,有助于我们更好地理解不同文化传统中建筑类型的形成机制。

\textbf{(3)丰富"形式-意义"关系的理论认识}

传统类型学研究多关注"形式→意义"的直接对应关系,本研究则揭示了"形式→规则→意义"的间接对应路径。对于规则驱动型建筑而言,形式是规则的编码结果,理解意义需要还原规则。这一认识丰富了建筑学关于形式与意义关系的理论探讨,也为建筑符号学研究提供了新的分析维度。

\textbf{(4)推动中西建筑理论的对话与互鉴}

本研究的理论建构过程,本身就是一次中西建筑理论对话的尝试。通过将罗西类型学与中国传统建筑经验相结合,本研究探索了一条"以西方理论为参照、以中国经验为基础、形成本土化理论创新"的学术路径。这种对话与互鉴的研究范式,对于中国建筑学的理论建设具有方法论意义。

\subsubsection{方法意义}

\textbf{(1)构建多学科交叉的综合分析框架}

本研究整合建筑学、符号学、人类学与计算机科学等多学科的方法资源,构建适用于规则驱动型建筑的综合分析框架。这一框架打破了传统遗产研究中"形态分析"与"文化阐释"相互割裂的局面,实现了定量分析与定性理解、视觉识别与语义推理的有机结合。这种跨学科整合的研究范式,对于建筑遗产研究方法的创新具有示范意义。

\textbf{(2)发展空间句法的"身份维度"}

空间句法是建筑空间分析的重要工具,但其经典模型预设了"无身份的行人",难以处理社会规则对空间使用的约束。本研究在此基础上引入"身份维度",发展"伦理句法"的分析视角,将分析焦点从"物理可达性"拓展到"伦理许可性"。这一方法拓展使空间句法能够更好地适用于礼制建筑、宗教建筑等社会规则约束明显的建筑类型,丰富了空间句法的应用范围。

\textbf{(3)实现隐性知识的显性化与可计算化}

传统建筑遗产中蕴含着大量隐性的民俗知识与营造智慧,这些知识往往存在于匠人口传心授的实践传统中,难以被系统记录与传承。本研究通过知识图谱与本体建模技术,将分散于族谱、方志、礼制文献中的隐性知识进行结构化表达,使其成为可检索、可推理、可传承的显性知识资源。这一工作为传统建筑知识的数字化保存与活化利用提供了方法路径。

\textbf{(4)探索人工智能赋能遗产研究的新模式}

当前,人工智能技术在文化遗产领域的应用多停留在图像识别、三维重建等技术层面,与遗产研究的理论深度尚存在差距。本研究将多模态人工智能技术与建筑类型学理论深度整合,构建"语义分割→知识图谱→RAG推理"的分析流程,探索人工智能赋能遗产研究的新模式。这种"理论引导技术、技术服务理论"的整合路径,对于推动人工智能与人文学科的交叉融合具有探索价值。

\subsubsection{实践意义}

\textbf{(1)服务国家文化遗产保护战略}

党的二十大报告明确提出要"加大文物和文化遗产保护力度"。岭南传统建筑是中华优秀传统文化的重要载体,其保护传承工作具有重要的文化价值与社会意义。本研究构建的分析方法与知识库,可为岭南传统建筑的价值认知、保护规划与活化利用提供科学依据,直接服务于国家文化遗产保护战略的实施。

\textbf{(2)支撑文化遗产数字化建设}

《关于推进实施国家文化数字化战略的意见》明确提出要"推动文化遗产数字化保护"。本研究基于CIDOC CRM国际标准构建岭南传统建筑知识库,实现了与国际遗产信息体系的语义兼容,为文化遗产数字化建设提供了可参照的实践案例。研究形成的技术流程与数据规范,可为其他地区传统建筑的数字化工作提供借鉴。

\textbf{(3)促进中华优秀传统文化的创造性转化与创新性发展}

习近平总书记强调,要推动中华优秀传统文化创造性转化、创新性发展。传统建筑遗产的保护传承,不仅在于物质形态的保存,更在于其所承载的文化内涵的阐释与传播。本研究通过多模态分析方法揭示广府宗祠背后的礼制逻辑与文化意蕴,将"看得见"的建筑形式与"看不见"的文化规则联系起来,有助于公众更加深入地理解传统建筑的文化价值,促进传统建筑文化的当代传承。

\textbf{(4)助力粤港澳大湾区文化建设}

粤港澳大湾区建设是国家重大发展战略,文化认同是大湾区融合发展的重要纽带。广府文化是粤港澳三地共同的文化根脉,广府宗祠作为广府文化的物质载体,见证了湾区人民共同的历史记忆与宗族情感。本研究对广府宗祠的系统研究与知识库建设,可为大湾区文化遗产的协同保护与文化认同的凝聚提供学术支撑。

\textbf{(5)为乡村振兴中的传统村落保护提供参考}

广府宗祠多分布于传统村落之中,是传统村落文化景观的核心要素。在全面推进乡村振兴的背景下,传统村落的保护与发展受到高度关注。本研究形成的理论框架与分析方法,可为传统村落中宗祠建筑的价值评估、保护策略制定与文化展示阐释提供参考,服务于乡村文化振兴事业。